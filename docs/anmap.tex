\documentstyle[12pt]{article}
%
% Latex Macros for use with Anmap
%
% The Anmap Name
\newcommand{\Anmap}{{\bf Anmap}}
\newcommand{\Version}{{\bf X0.52}}
%
% Page layout suitable for an A4 printer
\topmargin 0.1in
\oddsidemargin 0.1in
\evensidemargin 0.1in
\textheight 9.0in
\textwidth 6.25in
\headheight 0in
\headsep 0in
\parindent 0in
\parskip \bigskipamount
%
%
%
% Definition of Standard LATEX environments etc.
%
% Environment definitions for numbered paragraphs:
%   npars:-      (1) ....
%   rpars:-      (i) ....
%   apars:-      (a) ....
%
\newcounter{ncount}
\newcounter{rcount}
\newcounter{acount}
\newenvironment{npars}{
   \begin{list}{(\arabic{ncount})}{\usecounter{ncount}}
                      }{\end{list}}
\newenvironment{rpars}{
   \begin{list}{(\roman{rcount})}{\usecounter{rcount}}
                      }{\end{list}}
\newenvironment{apars}{
   \begin{list}{(\alph{acount})}{\usecounter{acount}}
                      }{\end{list}}
%
%
%


\begin{document}
\begin{center}
{\large\bf Anmap --- An image and data analysis system}
\end{center}

Anmap is a system for the analysis and processing of images
and spectral data.  Originally written for use in Radio Astronomy,
much of its functionality is applicable to other disciplines and
the package is being enhanced by the addition of new algorithms
and analysis procedures of direct use in, for example, NMR
imaging and spectroscopy.  The unique feature of Anmap is the
emphasis placed on the analysis of data with an aim to extracting
quantitative results for comparison with theoretical models
and/or other experimantal data.  To achieve this Anmap provides:
(a) A wide range of tools for analysis, fitting and modelling
(including standard image and data processing algorithms);
(b) A powerful environment for users to develop their own
analysis/processing tools either by combining existing 
algorithms and facilities with the very powerful command
(scripting) language or by writing new routines in FORTRAN
which integrate seemlessly with the rest of Anmap.  

{\bf Features of Anmap include the following:}
\begin{npars}
\item A powerful interface and development environment
which is ideally suited to both new and advanced users
as well as developers in the Anmap Environment.
\begin{rpars}
\item Anmap's functionality is accessed by
a user-friendly interface mixing both descriptive
commands and informative prompts for parameters with an
X-windows based graphical user interface (GUI). These modes
of operation are available {\em simultaneously to the user}.
\item On-line documentation and help available from the command
line or GUI.
\item Built in tools such as: editor; calculator; 
clipboard; file manager; options editor.
\item Ability to access all standard UNIX commands as if
part of the package --- Anmap appears much like a UNIX
shell.
\item A sophisticated command language with a full range of
language features: variables; arrays; lists; loops; 
conditionals; procedures; event and background processing; 
interprocess communication; access to data,
header information etc.
\item The GUI is fully programmable from the command language,
users may create their own windows with the full functionality
of the most powerful X-window based toolkits!
\item Easy writing and integration of FORTRAN code to add new
functionality, a large range of routines are provided for
data manipulation and analysis, I/O, user interaction etc.
Routines added in this way are indistinguishable from built-in
Anmap functionality.
\end{rpars}
\newpage
\item Analysis and processing algorithms cover both standard
image/data processing and specialised modelling and analysis
with error estimation where applicable.
\begin{rpars}
\item Full range of routines for manipulation of
image and spectral data: arithmetic operations; Fourier
Transforms; filtering; edge-detection; convolution, statistics; 
integration; slices; measurements from images.
\item Image deconvolution via the CLEAN algorithm.
\item Image editing and modelling: pixel-based image editing;
region and slice extraction; construction of model images and
masks; image reprojection, rotation, flips and re-sampling.
\item Model fitting of both spectral and image data: routines
applicable to both Radio Astronomy and NMR.
\item Specialised analysis routines: fractal dimensionality of
images; cluster analysis; percolation analysis.
\item Full range of routines for analysis of multi-frequency
polarization Radio Astronomical data.
\item Anmap can handle mathematical equations (for example for
plotting functions), expressed in a natural way such as
$sin(x) exp(-(x-x0)**2/2s)$.
\end{rpars}
\item Data are stored in an internal Anmap format with the
ability to import and export data from a number of formats.
Data within Anmap can be organised using a catalogue (simple
data base) system.
\begin{rpars}
\item import/export directly from/to NMR image and spectral files, 
FITS, raw ascii and binary data;
\item or via external filters to/from GIF, TIFF, RLE.
\item Handle multiple catalogues to organize work; data are stored
in standard UNIX files within the normal directory structure.
\item  Output images created automatically and added to the
catalogue without need for users to continually think up
file names (no more junk1.dat temp.dat!).
\end{rpars}
\item Graphics functionality combines
excellent publication quality output with data display.
\begin{rpars}
\item Object-oriented graphics system with control of
picture composition.
\item Scientific graphs with: multiple lines each with
individual line styles, colour and markers; automatic
keys; full control over style of frames titles etc.;
log-log, log-linear, linear-linear plots; automatic scaling and
peak matching; histograms; function plotting; annotations.
\item Image display via: grey-scale/colour images; contour
plots; surface plots; isometric plots; vector fields;
manipulation of colour pallets; annotations; symbol plots.
\item Drawing system and various other graphical output
from analysis routines can all be combined in one
publication plot exported as a postscript file.
\item Screen capture in a variety of formats.
\end{rpars}
\end{npars}

Anmap is still under development and a number of important
enhancements are planned before I consider it to be 
ready for general use:
\begin{apars}
\item The visualization and graphics in general will
be revised and improved.  Specifically the program will
be converted to use a 3D package capable of good visualization
such as PHIGS (or other PEX API) or OpenGL.
\item Suport for N-dimensional data will be added and existing
algorithms adapted where necessary.
\item The use of some proprietry software (mainly NAG) needs
to be revised.
\end{apars}
\vspace{1.5cm}
Dr. Paul Alexander \hfill October 1993
\newline
Mullard Radio Astronomy Observatory,
\newline
Cavendish Laboratory, Cambridge, CB3 OHE.
\end{document}


