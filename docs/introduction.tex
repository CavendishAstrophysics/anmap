\documentstyle[12pt]{article}
%
% Latex Macros for use with Anmap
%
% The Anmap Name
\newcommand{\Anmap}{{\bf Anmap}}
\newcommand{\Version}{{\bf X0.52}}
%
% Page layout suitable for an A4 printer
\topmargin 0.1in
\oddsidemargin 0.1in
\evensidemargin 0.1in
\textheight 9.0in
\textwidth 6.25in
\headheight 0in
\headsep 0in
\parindent 0in
\parskip \bigskipamount
%
%
%
% Definition of Standard LATEX environments etc.
%
% Environment definitions for numbered paragraphs:
%   npars:-      (1) ....
%   rpars:-      (i) ....
%   apars:-      (a) ....
%
\newcounter{ncount}
\newcounter{rcount}
\newcounter{acount}
\newenvironment{npars}{
   \begin{list}{(\arabic{ncount})}{\usecounter{ncount}}
                      }{\end{list}}
\newenvironment{rpars}{
   \begin{list}{(\roman{rcount})}{\usecounter{rcount}}
                      }{\end{list}}
\newenvironment{apars}{
   \begin{list}{(\alph{acount})}{\usecounter{acount}}
                      }{\end{list}}
%
%
%


\begin{document}
\begin{center}
\vspace*{4.0cm}
{\Large\bf Anmap User Guide} \\
\vspace{1.0cm}
{\Large\bf An Introduction to Anmap}
\end{center}

\newpage
\begin{center}
{\Large\bf Anmap User Guide} \\
\vspace{1.0cm}
{\Large\bf An Introduction to Anmap}
\end{center}
\vspace{1.0cm}

\Anmap\ is a system for the analysis and processing of images
and spectral data.  Originally written for use in Radio Astronomy,
much of its functionality is applicable to other disciplines and
the package is being enhanced by the addition of new algorithms
and analysis procedures of direct use in, for example, NMR
imaging and spectroscopy.  The unique feature of \Anmap\ is the
emphasis placed on the analysis of data with an aim to extracting
quantitative results for comparison with theoretical models
and/or other experimantal data.  To achieve this Anmap provides:
(a) A wide range of tools for analysis, fitting and modelling
(including standard image and data processing algorithms);
(b) A powerful environment for users to develop their own
analysis/processing tools either by combining existing 
algorithms and facilities with the very powerful command
(scripting) language or by writing new routines in FORTRAN or C
which integrate seemlessly with the rest of Anmap.  
 
Throughout this document some standard notation will be used.
\begin{apars}
\item In the text italics will indicate a command name or parameter.
\item In examples bold will indicate messages produced by the
computer as well as prompts.
\item The term CR will refer to a 'carriage-return' (i.e.
pressing the return key).
\end{apars}

\section{Getting \Anmap\ started}
 
To start up ANMAP simply type one of the following
\begin{tabbing}
XXXXXXX \= SUB-SYSTEMS \= \kill
\> anmap \\
\> Anmap \\
\> Xanmap \\
\end{tabbing}
where the final one of these will start up the X-windows version of
the program.

The program will then begin -- it will take a little time to
get going as the program is quite large and it is going
through a start-up procedure. After a while a message similar
to the following will appear on the screen:
\begin{verbatim}
         Welcome to Version X0.51e See Help/News for details
         =>      Bug fix in Multi-Add and Multi-Fit
         =>      See News about hopeful/possible fix of STACK problems !!?
         =>      See News concerning use of PGPLOT special characters
\end{verbatim}
and the prompt will change to:
\begin{tabbing}
XXXXXXX \= SUB-SYSTEMS \= \kill
\>  Anmap$>$ \\
\end{tabbing}
If you start up the X-windows version then a new command input
window with menu bar will appear together with a graphics window a
the message will be displayed above the menu bar
Anmap is now ready to accept commands. 

\section{The User Interface}

The facilities of \Anmap\ can be acessed either from the
command line or via a windows interface although in the
current verion \Version\ only a limited subset of facilities
currently have a windows interface.  The power of \Anmap\ stems
from the user's abaility to extend the basic functionality of
the program by either writing new routines (and window's interfaces)
in the command language used by \Anmap\ called Tcl, or by adding
new routines written in FORTYRAN or C.

Commands come in three flavours, user commands which implement the
analysis features of \Anmap\, Tcl (or command language) commands
which provide the basic operations of any computer language together
with many specific commands for data processing, and finally
ordinary UNIX-type commands and programs that you would use from
any unix command interface, SHELL --- \Anmap\ is in most ways just
another UNIX shell eextended with data processing commands.
New users will start just be using the {\em user} commands, 
but you will soon find that using the command language helps 
you to get your work done even faster!  The three flavours of command 
are distinguished simply:
\begin{npars}
\item User commands are all {\em multi-part} commands with
dashes (or hyphens / minus signs, ``-'') used to separate the parts of the
command --- a user only has to type sufficient of the command to
make it unambiguous for the command to be executed with each part
of the command being matched separately (see below).
\item Command language commands are single word commands, or
a multi-word command with the words separated by an underscore
character (``\_'') --- these commands must be typed exactly, minimal
matching is not allowed.
\item Finally, if the input you type cannot be interpreted as either
a user analysis command or a command language command then \Anmap\
attempts to execute what you have typed as a standard UNIX command
(program) in the same way as any command you type at a UNIX prompt.
\end{npars}
 
The conventions used for the mininal matching of user commands
are as follows:
Each command consists of a number of parts each separated by a
dash '--', with commands being case sensitive (ie. upper and lower-case 
characters {\em are} distinguished in commands as they are in
file names). The command
interpreter matches each part of the command separately, you
only need to type as much of the command as is necessary to
make it unambiguous. The simplest way to see how this works is
via an example. The command {\em list-catalogue} lists the contents
of the map-catalogue (see below). To use this command the user
may type any one of the following abbreviations:
\begin{tabbing}
XXXXXX\= \kill 
\>li-cat \\
\>list-c \\
\>li-c \\
\end{tabbing}
however, the simple {\em l-cat} 
is not sufficient as there is also a command {\em load-catalogue}
and a message:

{\bf *** command is ambiguous}

 will appear along with a list of alternatives.
 
User commands may take data: the data for a
particular command is given after the command --- you may type on
the same line as the command the responses to the prompts
which would normally follow after specifying the command name.
If more than one data item or response is given on the command
line than they may be separated by commas ``,'' or spaces. If
you do not know what data a particular command is expecting
then the program will give an informative prompt.

Commands can be continued onto more than one line by making the last 
character on the line a backslash \\.

{\bf Example 1}
The following will all produce the same result --- listing the first
ten entries in the catalogue
\begin{tabbing}
XXXXXXX \= SUB-SYSTEMS \= \kill
\> Anmap$>$ {\bf list-catalogue 1,10} \\
\\
\> Range to list low  [ 1 ] : {\bf 1} \\
\> Range to list high  [ 256 ] : {\bf 10} \\
\end{tabbing}

{\bf Example 2} 
Scan-map takes a map-catalogue entry and a UV-range -- in this
example the default of the complete map is used.
\begin{tabbing}
XXXXXXX \= SUB-SYSTEMS \= \kill
\> Anmap$>$ {\bf map-analysis scan-map 1,,,,,}
\end{tabbing}


When you are prompted for some input the default to the
command is shown in square brackets ``['', ``]''. Round brackets
are used to show possible responses. The default may be
obtained by pressing CR.

{\bf Example 3}
In this example a default directory and file type are
indicated -- if you do not supply these the defaults are used.
\begin{tabbing}
XXXXXXX \= SUB-SYSTEMS \= \kill
\> Map-name [/data1/pa/images/*.map] : {\bf CR} 
\end{tabbing}


{\bf Example 4}
The user is prompted for the projection of the output map in
the reproject command, the valid replies are either
equatorial, sky or tangent and the default (obtained on
pressing carriage return) is sky, in this example the user has
chosen equatorial (note again the use of minimal matching for matching
list-type output.
\begin{tabbing}
XXXXXXX \= SUB-SYSTEMS \= \kill
\> Projection (equatorial,sky,tangent) [sky] : {\bf equat}
\end{tabbing}
 
Command language commands provide access to a wide range of programming
tools as well as low-level access the \Anmap\ itself.  The command
language is based on Tcl an embedded language developped by
Dr. J Ousterhout at the University of California Berkely.  \Anmap\
is therefore built around a very powerful programming language which
in addition to all of the normal programming tools also allows you to
construct graphical user interfaces based on the X-windows system --- a
programmer using the \Anmap\ command language can therfore create a
complete X-windows application inside of \Anmap!
Command language commands cannot be abreviated and in general the data
to the commands must be specified on the command line using a UNIX-like
syntax of command value -options.  Examples of some of the most useful
command-language commands with a brief description of what they do are
given in table 1.
\begin{table}
\begin{tabular}{|l||l|}
\hline
& \\
{\bf Command(s)} & {\bf What is does} \\
& \\ \hline
& \\
set & set variables arrays and lists \\
foreach, loop, for & looping commands \\
proc, return, rename & manage procedures \\
if, switch & logical tests \\
open, close, file, puts, gets & access text files \\
exec, system, pid & run UNIX commands \\
string, regexp, regsub & handling text \\
list ... & handling lists of elements \\
expr, incr & calculations \\
& \\ \hline
\end{tabular}
\caption{Examples of command language commands}
\end{table}
A more detailed description of the command language is given
later.

\section{How \Anmap\ is organised}

\Anmap\ is organized into
sub-systems, each with its own set of commands, however some
commands (Global-Commands) are always available to you --- the
global commands are those which are available when you see the
``Anmap$>$'' prompt and you may type them in response to any of the
other prompts. Global-commands exist to obtain help
information, access the map-catalogue, access disc files and
importantly to change to another sub-system.  Note that all
command-language (Tcl) commands are treated as global commands.
The sub-systems which are currently defined are shown in table 2.
\begin{table}
\begin{tabular}{|l||l|}
\hline
& \\
{\bf Sub-system} & {\bf What is does} \\
& \\ \hline
& \\
graphic-system & control graphics and graphic devices \\
map-display & display of maps/images (grey scales contours etc.) \\
data-display & graphs, scatter plots etc. \\
scratch-display & control scratch graphics \\
drawing-display & annotations for graphics \\
map-analysis & analyse maps and images \\
spectrum-analysis & analyse spectral files \\
edit-image & edit and filter images \\
edit-redtape & edit the image/map headers \\
clean-system & clean image deconvolution \\
catalogue-system & manage the image catalogue \\
synchrotron-analysis & analyse maps of synchrotron emission \\
& \\ \hline
\end{tabular}
\caption{The main \Anmap\ sub-systems}
\end{table}

 
You start off in the main \Anmap-system - to return to it from any
other sub-system just type exit or quit --- these commands
typed in response to the Anmap$>$ prompt will exit \Anmap. 
To switch to a different
sub-system simply give the name of the sub-system as a command
and you will notice that the prompt will change.
At any time you can list the available commands by typing a
``?'' .  One useful feature is that you do not have to switch sub-systems
to access the commands in another sub-system, instead you can prefix
the required command with the name of the sub-system.  For example
if you are in the map-display sub-system and want to clear the screen
you need the {\em clear-screen} command from the graphic-system and
you can simply type:
\begin{tabbing}
XXXXXXX \= SUB-SYSTEMS \= \kill
\> Map-Display$>$  {\bf graphic-system clear-screen}
\end{tabbing}
 
You may obtain considerably more information by typing {\em help} ---
this enters the help system, while {\em news} is a similar command
providing up-to-date information on what has changed recently in
\Anmap.  Both of these commands give you a list of the available
information, the information is organised hierarchically so that
you will get a new list of known information as you move through
the hierarchy or tree.
To obtain information on a
particular topic in help or news, for example the catalogue-system, 
simply type the name of the entry of interest. Let us suppose you type
cat-sys (note that minimal matching applies). On the screen
will be displayed a few lines of introduction to the map
catalogue and then there will appear a new menu of things help
knows about in connection with the map-catalogue and the
catalogue system; as we will
presume you are new to \Anmap\, type intro and the information
listed under the Introduction will be typed. On completion of
this information you will still be at the level of help
information on the map catalogue -- to re-display the menu
type a question mark (?). 
To return to the Anmap$>$ prompt keep pressing CR until
the prompt re-appears on your screen.
In addition to the command-line versions, if you are using Xanmap
then X-windows tools exist to navigate through the help, these
can be started from the menu by selecting {\em Anmap Help}
or {\em Anmap News} from the {\em Help} menu.
 
 
\section{Interupting Anmap Commands}
 
You may interrupt the program at any stage by pressing the 
Control-Y i.e. the control and the "Y" keys simultaneously.
This should return you to the current sub-system prompt.
Note that this will happen immediately in response to a
prompt, but there may be a considerable delay if some work is
being done on an image as the effect is carefully controlled
so that you do not lose any data.  You can abort \Anmap\
completely by using the standard UNIX Control-C or choosing 
kill from the frame menu if it is available under X-windows.



\section{Finding Out About Commands}

A help facility is provided to explain the various commands
available which can be used by just typing {\em help}.
This is similar to the help system available on DEC VAX computers.
A list of topics help knows about will appear on the
screen and the prompt ``topic$>$''. To find out more about a
particular topic simply type the name of the topic required
(as with news described above). Some information will now be
printed on the terminal with possibly another set of options
--- note how the prompt changes so that you know where you are
in help. Help is organized in a hierarchical way, it is
probably worth spending some time exploring help now before
doing anything else. To exit help simply press return a number
of times --- this moves you back up through the help tree, or
type Control-Y to return immediately to the Anmap$>$ prompt.
 
In addition to using help you can obtain a list of all the
current commands and a brief indication of what they do simply
by typing a question mark ``?'', while
\begin{tabbing}
XXXXXXX \= SUB-SYSTEMS \= \kill
\> Anmap$>$ {\bf ? delete}
\end{tabbing}
will list all commands beginning delete.

\section{Data files and Data formats}

\Anmap\ uses at present an internal format for maps (images) and
a very simple format for spectral-type data (i.e. data which is
represented as values along a specified axis).  To read image data
into \Anmap\ you must either have your data already in the \Anmap\
format or use the {\em read-data-file} command (or perhaps another
specific data input command such as one of the commands to
read nmr data files: {\em nmr-read-image}, {\em nmr-image-read} 
or {\em nmr-serial-read}) to read data into the map catalogue.  
The {\em read-data-file} command can be used to read data in any
of the following formats:
\begin{npars}
\item FITS format files.
\item Ascii files with one value per pixel
\item Binary files with the data stored as REAL or INTEGER data.
\end{npars}
Other MRAO software, specifically POSTMORTEM, will however write data
in the same format as use by \Anmap\ and in this case the maps can
be added directly to the map-catalogue using the {\em add-to-catalogue}
command.
Once the map/image data are in the catalogue a number of commands can
be used to manipulate them, the commands you are most likely to use
to start with are shown in table 3.
\begin{table}
\begin{tabular}{|l||l|}
\hline
& \\
{\bf Catalogue command} & {\bf What is does} \\
& \\ \hline
& \\
read-data-file & read data into \Anmap\ \\
add-to-catalogue & add maps to the catalogue \\
remove-from-catalogue & remove maps from thecatalogue \\
list-catalogue & list (a part of) the catalogue \\
export-permanent-map & save temporary maps \\
delete-catalogue-entry & delete maps from disc \\
& \\ \hline
\end{tabular}
\caption{The most common catalogue commands.  These commands are
all available in the map-catalogue sub-system although they are also
defined as global commands}
\end{table}

Spectral files have a very simple format and therefore it should be easy
to get your data into the required format, specifically the file is an
ascii (ordinary text) file of numbers where each column in the file is
a different axis --- for example data specified as $y$ as a function of
$x$ would have the $x$-values in column 1 and the $y$-values in column 2.
The only other information that is needed in the file is header information.
This must preceede all the columns of data and each line of header information
start with a \% symbol followed immediately by a name by which to
identify the header item then the value.  Files {\em must} contain two
items \%ndata and \%ncols giving the number of lines of data and the
number of columns in the file.  The following is a simple example of a
spectral file:
\begin{verbatim}
%ndata 10
%ncols 2
%title This is an example spectral file
1 0.5612
2 1.6540
3 9.0875
4 12.0241
5 9.0123
6 8.0135
7 6.1209
8 2.1218
9 0.0671
10 0.0056
\end{verbatim}
\end{document}
 
 
 
 
 
 
 
 
 
 
 
 
 
 
 
 
 
 
 
 
 
 
 
 
 
 
 
 
 
 
 
 
 
 
 
 
 
 
 
 
 
 
 
 
 
 
 
 
 

