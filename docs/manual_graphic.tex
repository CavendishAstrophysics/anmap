\documentstyle[12pt]{article}
%
% Latex Macros for use with Anmap
%
% The Anmap Name
\newcommand{\Anmap}{{\bf Anmap}}
\newcommand{\Version}{{\bf X0.52}}
%
% Page layout suitable for an A4 printer
\topmargin 0.1in
\oddsidemargin 0.1in
\evensidemargin 0.1in
\textheight 9.0in
\textwidth 6.25in
\headheight 0in
\headsep 0in
\parindent 0in
\parskip \bigskipamount
%
%
%
% Definition of Standard LATEX environments etc.
%
% Environment definitions for numbered paragraphs:
%   npars:-      (1) ....
%   rpars:-      (i) ....
%   apars:-      (a) ....
%
\newcounter{ncount}
\newcounter{rcount}
\newcounter{acount}
\newenvironment{npars}{
   \begin{list}{(\arabic{ncount})}{\usecounter{ncount}}
                      }{\end{list}}
\newenvironment{rpars}{
   \begin{list}{(\roman{rcount})}{\usecounter{rcount}}
                      }{\end{list}}
\newenvironment{apars}{
   \begin{list}{(\alph{acount})}{\usecounter{acount}}
                      }{\end{list}}
%
%
%


\begin{document}
\begin{center}
\vspace*{4.0cm}
{\Large\bf Anmap User Guide} \\
\vspace{1.0cm}
{\large\bf to} \\
\vspace{1.0cm}
{\Large\bf Commands for Publication Graphics}
\end{center}

\newpage
\section{Introduction}

Publication quality graphics in \Anmap\ are provided for graphs
and images in a large variety of styles with control over almost
every aspect of the output.  Complicated figures can be composed
using the tools provided which include commands to layout various
aspects of the figure, annotate them and even add simple line
drawings.  The graphics facilities are {\em object oriented}, this
means that you construct plots by building up a description and
then displying it, the power of this approach comes when you want to
change an aspect of the plot, for example the line style or text size,
all you have to do is alter these aspects of the plot and then re-display,
you do not have to repeat a perhaps long and complicated series of
actions to  re-display the figure.

This Users' Guide describes the command interface to the publication
graphics facilities of \Anmap; it will therefore be of interest to
all users wanting to display such graphics and users writing scripts
which nake use of the graphics facilities.  

The graphics-related commands are organised into five sub-systems.
The {\em graphics} sub-system provides basic commands to control the
layout of graphical objects on the page, specify basic attributes for
graphical objects and control the graphic devices which \Anmap\ can send
its output to.  Each of the other sub-systems provide facilities to
construct graphic objects of one of four basic types (image, graph,
drawings and scratch graphics in the {\em map-display}, 
{\em data-display}, {\em drawing-display} and {\em scratch-display}
sub-systems respectively).  Multiple objects of each type may also be
defined all with their own (different) set of attributes.  For each
graphic object (excluding drawings) you may also specify annotations
such as text, arrows etc.

In the {\em map-display} sub-system you construct plots (graphic objects)
of two-dimensional images --- maps.  You can display the images as 
grey-scales or in false-colour, or using contours, 
vectors (for appropriate data), or symbol plots to mark specified 
data points; these various styles can be combined on a single plot.
You have full control over the style of the plot including the frame,
grid, text annotations, contour line styles, scale-bar for the grey-scale
etc.

The {em data-display} sub-system is used to create plots of 
one-dimensional data or functions where a series of ``y-values''
are specified at a series of ``x-values''.  The data may either
form a continuous graph or histogram, or they could represent points
in a scatter plot.  The data points may be annotated with error bars
in both the ``x'' and ``y'' directions, and you have full control over
the appearance of the plot deciding on whether individual points are
marked with symbols, joined by a continuous curve or a histogram-type
curve and so on.  All aspects of the frae, titles and annotations may
be specified and a key generated automatically.  On a single plot
you can specify many ``lines'' and facilities exist to provide automatic
scaling (both in y and x for comparison of say spectra) and offsetting
in y and x directions to produce a stacked-plot.  Data can be supplied
either as a file in the standard format used by \Anmap\ (it is a simple
multi-column ascii (ordinary text) file, or as a mathematical
function given in a intuative form (such as $sin(x)/x$ if you wanted to
plot a {\em sinc} function).  

The final two sub-systems, the {\em scratch-display} sub-system and 
the {\em drawing-display} sub-system provide facilities to control
the style of scratch graphics and to annotate the entire figure 
respectively.  Scratch-graphics are those graphics produced by 
various commands within \Anmap\ such as isometric or surface plots,
scatter plots or histograms of image values.  The {\em scratch=display}
sub-system provides facilities for specifying the size of such plots,
line- and text-styles.  Although each graphics sub-system provides
a set of annotation commands, there is also a {\em drawing-display}
sub-system to complement these facilities.  The annotations associated
with a particular graphic object are tied to that object (and indeed
use the same coordinate system as the object).  When you compose
a complete figure out of a number of graphic objects (for example a
number of different images), you may well wish to annotate the complete
composition perhaps with a title, or lines or arrows linking
various graphic objects; these facilities are provided for in the
{\em drawing-display} sub-system.

The remainder of this Users' Guide is organised as follows.  In the
next section commands are described which operate on the complete figure, 
or which are common to all subs-systems.  An example is then given
illustrating the basic principles of the object-oriented graphics
approach and then a more detailed description of the facilities 
available within each sub-system is given.  The aim of this guide is
to show you how to use the publication graphics, it is not an exhaustive
list of all the facilities or options available and you should consult
the on-line help documentation for a complete description of all of
the available commands.

\section{Basic Graphics Commands}

There are a number of commands which are available from all the
graphic-based sub-systems within \Anmap\, the most important of these
are the {\em plot} and {\em initialise} commands.  Since the graphic's
system is {\bf object-oriented} you construct graphics by specifying
options and when you have a complete set of options defined then you
can give a plot command.  The {\em plot} command itself requires you
to specify one of a number of options which differ between sub-systems,
but all sub-systems have at least the options {\em all} and {\em refresh}.
Similarly the {\em initialise} command requires an option and all
sub-systems support {\em all} and {\em plot}.  A simple example will
illustrate the use of these commands.

In the {\em map-display} sub-system we can can setup to display a
grey-scale representation of an image in the map-catalogue as
follows:
\begin{tabbing}
XXXXXXX \= SUB-SYSTEMS \= \kill
\> Map-Display$>$ \> {\bf set-map 3} \\
\> Map-Display$>$ \> {\bf grey on 0.0 0.15} \\
\> Map-Display$>$ \> {\bf plot grey} \\
\end{tabbing}

Nothing appears on the graphic device until the plot command is given. 
If you were to have said {\em plot all} then in addition to the grey-scale
you would have a frame, title and other annotations plotted.\footnote{The
exact output will depend on the default options specified in the system,
or your own, initialisation file}  This sequence of commands will have
drawn a scale-bar on the grey-scale with text indicating the range displayed;
to turn off the text on the scale bar you can use the following command
to change one of the many options for the scale bar:
\begin{tabbing}
XXXXXXX \= SUB-SYSTEMS \= \kill
\> Map-Display$>$ \> {\bf set-style scale-bar text-off} \\
\> Map-Display$>$ \> {\bf plot refresh} \\
\end{tabbing}
Here the {\em plot refresh} command will re-draw the grey-scale image,
but with the new option (turning off text on the scale-bar) applied. 
The command {\em plot all} would not have had any effect since it is
an instruction to plot any outshanding graphic elements not yet plotted,
whereas {\em plot refresh} is an instruction to {\bf update} the current
displayed graphic.  To illustrate these differences let us take the 
above example a step further by adding contours to the plot.  We could
try to get contours plotted using the following commands:
\begin{tabbing}
XXXXXXX \= SUB-SYSTEMS \= \kill
\> Map-Display$>$ \> {\bf linear-contours 0.02,0.02,0.14} \\
\> Map-Display$>$ \> {\bf plot refresh} \\
\end{tabbing}
This will have defined 7 contour levels, but the {plot refresh}
command will not have caused them to be plotted, instead the grey-scale
would have been re-displayed since the contours are a new element to the
plot.  To get the contours plotted we must say {\em plot contours}
({\em plot all} would also display the contours but add a frame etc, as well),
therefore the correct sequence would have been:
\begin{tabbing}
XXXXXXX \= SUB-SYSTEMS \= \kill
\> Map-Display$>$ \> {\bf linear-contours 0.02,0.02,0.14} \\
\> Map-Display$>$ \> {\bf plot contours} \\
\end{tabbing}
and subsequent {\em plot refresh} commands would plot a grey-scale
overlaid with contours.  Although this is a very simple example it
serves to illustrate the way the graphic environment within \Anmap\
is organised.

In addition to the basic commands available within all graphic-sub-systems,
their are the important commands in the {\em graphic} sub-system
itself which are used to control the overall appearance of the display
and to construct complex graphics.  It is in the {\em graphic} sub-system
that the object-oriented approach to the graphical system within
\Anmap\ becomes clear since all the available commands act on complete
graphical objects.  The commands types above to the {\em map-display}
sub-system are in fact specifying options to a graphic object of type
{\bf image} and the plot commands are performing operations to display
graphics of type {\bf image}.  Each graphical sub-system has a different
type of graphics object as listed in Table 1.
\begin{table}
\begin{tabular}{|l||l|}
\hline
& \\
{\bf Object type} & {\bf sub-system} \\
& \\ \hline
& \\
image & map-display (contours, grey-scales, vectors etc.) \\
& \\
graph & data-display (graphs, histograms, scatter-plots etc.) \\
& \\
draw & drawing-display (drawings: lines, circles, text, etc.) \\
& \\
scratch & scratch-display (various graphical output) \\ 
& \\ \hline
\end{tabular}
\caption{Graphical objects available.  Commands in the
{\em graphic} sub-system control the properties of these objects.}
\end{table}

The {\em select} command selects a particular graphical object to
which the {\em graphic} sub-system commands apply.  You may have
more than one graphical object of a given type and they are
distinguished by specifying an integer key; when \Anmap\ is
initialised one graphical object of each type is defined and is
given the numerical key of ``1'' (unity).  Therefore, to select the
{\bf image} object we have already used as an example we would type
the following:
\begin{tabbing}
XXXXXXX \= SUB-SYSTEMS \= \kill
\> Graphic$>$ \> {\bf select image 1} \\
\end{tabbing}
or to start a new {\bf image} object (retaining the original for future
use):
\begin{tabbing}
XXXXXXX \= SUB-SYSTEMS \= \kill
\> Graphic$>$ \> {\bf select image 2} \\
\end{tabbing}
where in fact the ``2'' could be any integer key chosen by us for
convenience.\footnote{Supplied procedures use graphical objects with
keys above 100, you should avoid using these or confusing effects may
result, especially for novice users.}

Having selected a particualr graphical object all subsequent {\em graphic}
commands are applied to this object.  The most important commands used
to alter the properties of a particular graphic are listed in Table 2.
\begin{table}
\begin{tabular}{|l||l|}
\hline
& \\
{\bf Command} & {\bf Action} \\
& \\ \hline
& \\
view-port & specify the view-port into which the graphic is drawn \\
& parameters: $x1$,$x2$,$y1$,$y2$ (normalised device coordinates) \\
& \\
opaque &  specify an opaque rendering for the graphic, \\
& any graphics below this object cannot be seen \\
& (default for {\bf image}, {\bf graphic} and {\bf scratch} objects \\ 
& \\
transparent & specify a transparent rendering for the graphic, \\
& graphics below this object will show through \\
& (default for {\bf draw} objects \\
& \\ \hline
\end{tabular}
\caption{Graphic sub-system commands used to alter the properties of
graphical objects.}
\end{table}

\section{A Tutorial Example}

\section{The Map-Display Sub-System}

The {\em map-display} sub-system allows for very sophisticated 
image display using in a variety of formats and styles:
\begin{npars}
\item Grey-scale/colour plots
\item Contour plots
\item Surface relief plots
\item Vector fields
\item Symbol display of specific pixels
\end{npars}
The following sections discuss the commands which are available in the
{\em map-analysis} sub-system.

\subsection{set-map}

\Command{set the catalogue entry for display/plotting}
\Parameters{map-catalogue-entry}

The map to be displayed/plotted is specified with the 
{\em set-map} command.
In addition to specifying a map to display the current environment
is re-initialised so that a new plot will be produced; this is
usually equivalent to "initialise plot" and will lead the standard
options unaltered. If you want to overlay a map onto the current plot,
use the overlay-map command.

The following sequence will plot catalogue entry 1 then catalogue entry
2 in the current map-display window.
\begin{tabbing}
XXXXXXX \= SUB-SYSTEMS \= \kill
\>Map-display$>$ \> {\bf set-map 1}
\>Map-display$>$ \> {\bf grey on ,,,,}
\>Map-display$>$ \> {\bf plot all}
\>Map-display$>$ \> {\bf set-map 2}
\>Map-display$>$ \> {\bf grey on ,,,,}
\>Map-display$>$ \> {\bf plot all}
\end{tabbing}

When the set-map command is executed a procedure definition
new-map-initialise is run to (re-)initialise the required
options.  The default action is, for example, to do an
effective "initialise plot" command and also to re-initialise
all contour levels.  This action can be changed by re-definining
the new-map-initialise procedure to execute a procedure of your
own -- use the ^list-procedures command to see which procedure
file is executed by this command and copy it to your own file space,
edit this file (usually called new\_map\_initialise.com) and then
re-define the procedure new-map-initialise to execute this file.
For example you could put the following line in your init_anmap.tcl
file:
\begin{tabbing}
XXXXXXX \= SUB-SYSTEMS \= \kill
\> proc procedure new\_map\_initialise { } {
\> \> init all  
\> }
\end{tabbing}

\subsection{set-uv-range}

\Command{set the UV-range of the current map to display}
\Parameters{UV-range}

Specify the UV-range to display.  If you give this command
before any plot command then the specified UV-region will fill
the frame of the plot and the frame will be scaled so that
the U and V scales of the plot are equal.  After a plot command
the set-uv-range command will restrict the range of data plotted,
but the frame will be scaled to the UV-range which was in effect
at the time of the first "plot" command.


\subsection{set-pips}

\Command{control plotting of RA/DEC and/or UV pips}
\Parameters{radec-pip-on-off, [pip-interval], UV-pip-on-off}

Turn pips on/off and for the RA/DEC pips select the interval at
which to display the pips.

\subsection{set-grid}

\Command{control plotting of RA/DEC and UV grids}
\Parameters{radec-grid, [grid-ra,grid-dec], uv-grid, [grid-u,grid-v]}

Control plotting of RA/DEC and UV grids. If you specify on to either
you must also supply the interval for the grid:
\newline
minutes-of-time,minutes-of-arc   ---    for RA/DEC grid
\newline
pixels-u,pixels-v                ---    for UV grid
\newline
The default (obtained by typeing a 0) will plot approximately five
grid lines on the plot. To obtain, for example, a grid of the actual
pixel positions set radec-grid=off, uv-grid=on and grid-u=grid-v=1.

\subsection{set-interpolation}

\Parameters{on/off}

Control which routine is used for contouring -- by setting
interpolation on the map will be interpolated to mid-points
prior to plotting. This option can only be used if the region
of the map to be plotted is smaller than 256x256 pixels.

\subsection{Contours}
\Information{on contour levels and styles}

Anmap supports contours in five concurrent styles; the line-style actually
associated with each style is set using the "set-style contours line-style"
command.  The styles are labelled numerically from 1 to 5.

Three defaults are set up for positive, negative and zero contours -- the
positive contours are by default style 1, negative style 2 and zero style
3; style 1 is initialised to solid foreground, style 2 to dashed foreground
and style 3 to dotted foreground.

Three commands are available to set contour levels:
   contours style [list]                   specify a list of contours
                                           for style "style"
   linear-contours start increment end     specify a range of linear contours
                                           using the default positive,
                                           negative and zero styles.
   logarithmic-contours start factor end   specify a range of logarithmic
                                           contours using the default positive,
                                           negative and zero styles.

The styles actually associated with positive negative and zero default
styles can be set using the set-style contours command with the options
positive, negative and zero.  For example the following sets up some
line contours of style 1 and style 3:
   linear-contours 0.001,.001,.01
   set-style contours positive 3
   linear-contours .011,.001,.02
   plot contours

To plot contours use the "plot all" or "plot contours" commands; once
plotted each level is marked as plotted and will not be re-plotted, to force
the re-plotting of contour levels use the reset-contour-levels command.  To
clear the lists of contours use the clear-contour-levels command.  Both
the reset-... and clear-... take a parameter specifying either the particular
style upon which to act or ALL to specify resetting/clearing all the styles.

Contour levels are associated also with one of the main and overlay-maps, to
specify contours for these two maps then the appropraite map must be current;
to make the main and ovserlay maps current use the overlay-map command.


\subsection{linear-contours}

\Command{setup linear contours}
\Parameters{start-contour, interval, end-contour}

Specify linear contour levels as follows:


          level(n) = start-contour + (n-1) x interval


up to the "end-contour". Negative contours are added to the
list of contours of style 2, positive contours to the list of
style 1. To plot the contours on an existing frame use the
plot contours command.

If the set of contours you specify includes the zero-contour
(or one numerically very close) you will be prompted and asked
if you want to include it on the plot.

The default for the end-contour is the maximum on the map in
the specified UV-range: a map must be specified before you can
define linear contours for this reason (in fact an allowance
for the peak height increasing slightly when the data are
interpolated is made if you are using the interpolation option
-- the default).

\subsection{contours}

\Command{specify contours}
\Parameters{style [contour-list]}

Specify one or more contours of the specified style; the style is
selectec via specifying an integer in the range 1 - 5.  Contours
can be entered until you signify the end of the list by providing
a null entry (either an extra Carriage return or two or more commas).

Example:

Enter contours of style 2 (usually used for negative contours)

   contours 2, .001, .002, .004, .008, .012, .016, .02,,,

\subsection{logarithmic-contours}

\Command{set logarithmic contours}
\Parameters{start-contour, factor, end-contour}

Specify logarithmic contour levels as follows:

        level(n) = +/-  abs(start-contour x (factor(n-1)))

up to the "end-contour". Both positive and negative contours
are specified with negative contours added to the
list of contours of style 2 and positive contours to the list
of style 1. To plot contours on an existing frame use the plot
contours command.

The default for the end-contour is the maximum on the map in
the specified UV-range: for this reason a map must be specified
before you can define linear contours (see note under
linear-contours).


\subsection{reset-contour-levels}

\Command{reset the "plotted" flag for contours}
\Parameters{option [style]}

Reset the specified contour level list; This does not effect
the plot on the screen, but resets the contour lists so that
the lists have the same contour levels specified, but the
flag indicating they are "plotted" is removed; this enable the
contours to be "applied" to a new plot, but plot refresh will
not reproduce the current plot.  You can either specify "all"
or a particular style of contour.

\subsection{clear-contour-levels}

\Command{clear the "plotted" flag for contours}
\Parameters{option [style]}

Clear the specified lists of contours.  This does not effect
the plot on the screen, but all contour lists are cleared.
You can either specify "all" or a particular style of contour.

\subsection{grey-scale}

\Command{grey-scale or false colour display of current map}
\Parameters{on/off, low-level, high-level}

Turn grey-scaling/false-colour on or off and define the map values
to be mapped into the low level and the high level. On the TV device
the low-level will be black and the high level white whereas
on a hard-copy device the reverse scaling is used. The grey-
scale will be added to the same frame as the pips, grid,
contours and labelling with a go-plot command. The lookup-
table between pixel values and intensity levels is linear by
default. For the TV device the command modify-lookup-table can
be used to change the look-up table; linear, logarithmic,
colour power-law and inverse lookup-tables are supported.


The default for grey-scale is off.

{\large\em Example}
\newline
To plot a grey-scale with a labelled frame on the TV device.

      set-map 1
      set-uv-range 20,,,,
      grey-scale on, 0.0, 0.01
      set-style scale-bar on
      plot all

the scaling is from 0.0 to 0.01.

To plot a grey-scale with contours overlayed:

      set-map 1
      set-uv-range -10 10 10 -10
      grey-scale on, 0.0, 0.01
      logarithmic-contours .001, 2, .032
      plot all

To plot a grey-scale of one-map (2) superimposed on contours
of another (1) -- e.g. grey-scale of spectral index map on
I-map contours:

      set-map 2
      set-uv-range -10 10 10 -10
      grey-scale on 0.4, 1.5
      go
      overlay-map 1
      overlay-map 1
      grey-scale off
      log-contours .001, 2, .032
      plot contours

Note that the grey-scale is plotted before the contours --
this is necessary on certain devices, specifically Xwindows and the
the Laser-Printer, since the action of plotting
the grey-scale is to overwrite everything currently on the
screen. The order of the plot type commands is therefore
important in determining the final appearance of your plot,
commands defining the grey -scale and contour levels may be
given in any order -- of course the go-plot command gets the
order of plotting correct.

{\large\em lookup table}
\newline
The lookup table controls the way in which grey scales are represented
on the output device; each "grey-level" is mapped into a colour
representation -- this way of representing data is called false colour.
Anmap under UNIX does provide some support for true colour images which
use colour and intensity to encode information.  This is provdided in the
"true-colour" sub-system; the range of manipulations possible on true
colour images is however more considerably limited than the full range
of operations available for intensity images.

To change the lookup table for the main Anmap display you can use the
modify-lookup-table command.

\subsection{title-plot}
\Command{add a title to the plot}
\Parameters{title}

Add a title to the plot, plotted using the plot text command.


\subsection{crosses-file}

\Command{set options for displaying crosses}
\Parameters{option}
Options: on, off, file, size}

Crosses can be added to the plot using this command.  The crosses must
be specified in a file which has the following format:

A number of lines each one representing a cross with the following entry
per cross:

    RA             DEC            label

e.g.
    01 03 45.6     34 56 29.5     Source: 3C405

with RA and DEC refering to the cross position at the stanbdard epoch of
the coordinate system and label a label for the cross.

Options:

   on   --    turn plotting of crosses on
   off  --    turn plotting of crosses off
   file --    file name of the crosses file
   size --    size in UV of the cross to be drawn

All the crosses listed in the crosses file will then be drawn with
their respective labels.  The crosses will be drawn after a plot all
or plot crosses command.

\subsection{set-style}
\Command{set-style options for map-display}
\Parameters{option}
\Options{title frame contours grid symbol vectors annotations scale-bar reset}

This command enables a wide variety of "style" parameters for the map-display
sub-system to be set.  The "reset" options resets all options to their
default values.

{\em title}
\newline
Set options for titles, sub-options:

   on              turn plotting of title on
   off             turn plotting of title off
   text-style      set text-style for title (font, size, colour, width)

See also the title-plot and initialise title commands.

{\em frame}
\newline
Set options for the frame surrounding grey-scales/contour plots.
Sub-options:

   on              enable the plotting of the frame
   off             turn off the plotting of the frame
   text-style      set text style for markings on the frame
                    (font size colour width)
   line-style      set line-style for the frame itself
                    (style width colour)
   space           define an amount of space to leave around the frame
                   as a fraction of the view-port.  You may get greater
                   control of the size of the frame using the view-port
                   setting for the current image graphic (see help on
                   the graphic-systsem sub-system).
   pip-size        set the pip size for the frame
   pip-x-offset    set the offset of the labels from the right-hand edge
                   of the frame.
   pip-y-offset    set the offset of the labels from the bottom of the
                   frame

See also: set-pips set-grid "set-style grid"

{\em contours}
\newline
Set options for contours.  Sub-options:

   line-style  [style]     set the line-style for style "style" (style,
                           width, colour).
   positive-style [style]  set the default style for positive contours
   negative-style [style]  set the default style for negative contours
   zero-style [style]      set the default style for zero contours

"style" is an integer representing one of the five styles of contours.

{\em grid}
\newline
Set options for grid.  Sub-options:

   line-style      set the line-style for grid lines (style, width, colour).


{\em symbol}
\newline
Set options for symbol-text.  Sub-options:

   text-style      set the text-style for symbols (font, size, colourm width).

{\em vectors}
\newline
Set options for vectors.  Sub-options:

   line-style      set the line-style for vectors (style, width, colour).


{\em annotations}
\newline
Set options for annotations.  Sub-options:

  date-on        add date
  date-off       do not add date
  label-on       add information
  label-off      do not add information
  text-style     annotation text style (font, size, colour, width)
  text-drop      drop between lines (i.e. distance between successive
                 lines of text on multi-line text output).
  title-on       add title
  title-off      do not add title
  all-on         add everything
  all-off        do not add anything

The annotations controlled by this set-style option are those plotted using
the plot text option.  For general annotations see the help entry on
annotate.


{\em scale-bar}
\newline
Set options for the display of a scale-bar on grey-scales. Sub-options:

   on            turn on the display of the scale-bar
   off           turn off the display of the scale-bar
   width         set the width of the scale-bar as a fraction
                 of the frame size
   text-off      turn off adding an annotation to the scale bar
   text-on       turn on adding an annotation to the scale bar

{\em reset}
\newline
This option resets all style parameters to their default values (see also
initialise options).

{\em information attributes}
\newline
The following attributes may be set using the above style commands:

\subsection{initialise}
\Command{initialise the map-display sub-system}
\Parameters{option}
\Options{all plot options setup annotations title}

Initialise various aspects of the map-display sub-system.  The options
have the following meaning:

      all            initialise everything equivalent to all of the
                     other options
      plot           initialse flags for plotting; this sets the state
                     of the sub-system to one in which nothing has been
                     plotted (useful after re-setting a map).
      options        reset options and styles to their default values
      setup          initialise the setup of maps and overlay maps etc.
      annotations    initialise annotations and the annotation sub-system
      title          initialise the title of the plot; reset the title.

\subsection{vector-plot}


\subsection{symbol-plot}

\subsection{overlay-map}
\Command{define an overlay map}
\Parameters{option}
\Options{set off on current-main current-overlay}

The overlay-map command is used to control the display of two maps
for comparison purposes, the main and overlay maps.  Other options
also allow maps apart from the main map to be displayed, specifically
the vector and symbol options, but the overlay map option enables a
second map to be contours or grey-scaled.  All map-display commands
act on the current-map, the two options current-main and current-overlay
can be used to select the current map.   If an overlay map is selected
then a plot all command will plot everything for both the main and
overlay maps; this can be suppressed by selecting overlay-map off.

The overlay-map command itself takes one of a number of options:

   set                set the catalogue-entry for the overlay map, this
                      automatically turns the overlay-map option on and
                      makes the overlay map the dfault map.
   on / off           turn the overlay map option on/off; you can toggle the
                      overlay map action with these commands without
                      re-defining the overlay map itself
   current-main       make the main map current -- useful to enable you
                      to go back and add contour levels etc.
   current-overlay    make the overlay map current.

It is strongly suggested that you use the main map for grey-scales
and the overlay map for contours if this is the combination you wish to
have -- you may of course have both using contours.  This requirement
is needed so that "plot all" and "plot refresh" commands work properly
since the main map is plotted first.

The overlay-map restricts you to using just two maps on the same display;
in fact you can achieve very complex plots by just using "set-map" and
careful plot options, however you will have to give multiple commands
to refresh the plot rather than the single plot refresh command needed if
you make use of the overlay-map option.

{\em Example}
\newline
Do a grey scale of a spectral index map with intensity contours
overlayed:
  set-map alpha-map
  grey on 0.5,1.5
  overlay-map set intensity-map
  linear-contours .001,.001,.01
  plot all

In the previous example you decide you want to alter the grey-scale
of the main map, you can then type
  overlay-map current-main
  grey-scale on 0.45,1.25
  plot refresh

Or you would like just a grey-scale of the spectral index map
  overlay-map off
  plot refresh



\subsection{display}

This command enables various attribute settings to be
displayed. A number of sub-options are available:

{\em levels}
\newline
The currently defined contour levels are listed on the text
screen.

{\em all}
\newline
Display all information concerning options. This is equivalent
to the commands:

                   display options
                   display vectors
                   display symbols
                   display levels

{\em options}
\newline
Display options and maximum/minimum values for the current map
in the specified UV-range. Option settings and the settings of
the text and mark fonts/sizes etc. are written to the text
screen.

{\em vectors}
\newline
Display information on the current options defining the vector
plot.

{\em symbols}
\newline
Display information on the current options defining the symbol
plot.

This command enables various attribute settings to be displayed.
A number of sub-options are available:


\subsection{plot}
\Command{give a variety of plotting commands}
\Parameters{option}
\Options{all refresh frame grey contours 
text crosses vectors symbols annotations}

Issue a "plot" command to plot various parts of the display - to plot
everything use the plot all command and to rfresh the plot use the
plot refresh command.  Each option controls one aspect of the plot
except for the "all" and "refresh" options -- the "all" option will
plot all outstanding information while the "plot refresh" command
will refresh the current plot (applying any changes of style).
Plot followed by an option which is not "alL" or "refresh" will force
that aspect of the plot to be plotted.  For example suppose one does the
following:

   grey-scale on 0.0,0.005
   plot grey

The command "plot all" would then add a frame and text to the plot, but
would not replot the grey-scale, however a plot refresh would replot
the greyscale, but nothing more.  Consider this slightly different
example:

   plot frame
   grey scale on 0,0.05

Now a plot grey would plot the grey-scle on top of the frame,
obscuring part of it, you could then re-plot the frame with a
plot frame command; however a plot all command would just plot the
greyscale (and not replot the frame) and a plot refresh would just
replot the frame and not the grey scale.

An important use for plot refresh is to redraw the existing plot
either in a different frame or to a new device.

Options available are:

{\em frame}
\newline
The frame is plotted. The frame is either an RA/DEC astrometric frame
and/or a UV frame.  The frame may also have a grid (see set-grid).  The
style of the frame is controlled using the set-style frame command
and also the set-pips command to control the sorts of markings on
the frame.

{\em contours}
\newline
The currently defined contours will be added to the plot.  Only those
contours not already marked as plotted will be plotted, but this can
be changes using the reset-contour-levels command.


{\em grey}
\newline
The currently defined grey scale is plotted.  On almost all device the
greyscale is effectively opaque and therefore needs to be the first
thing drawn on the plot if, for example, contours and frame are to
be displayed properly.

A call to set-grey-scale will need to have been made to turn grey-scaling
on and to have defined the range.

{\em vectors}
\newline
The options defining the vector plot must have already been defined
using the vector-plot sub-system.  Vectors will then be added to the
current plot, it is therefore possible to overlay vectors on contour
or grey-scale plots.

{\em symbols}
\newline
The options defining the symbol plot are setup using the symbol-plot
sub-system and should already have been defined before this command is
used.  Symbols will be added to an existing plot or a new frame etc.
will be drawn and symbols plotted.

{\em text}
\newline
The text (title, date/history, contour-levels) will be plotted. Exactly
what is plotted can be controlled with the set-style annotations command.
The list of contours is a list containing just the contours plotted to
date -- the timing in the use of this command is therfore important.

{\em crosses}
\newline
Crosses from a crosses file will be drawn -- you need to have defined
necessary options for the crosses using the crosses-file command.

{\em all}
\newline
Plot all plots everything which is outstanding -- it is similar, bt
not strictly identical to the sequence of commands:

  plot grey
  plot frame
  plot contours
  plot text
  plot vectors
  plot symbol
  plot annotations

The difference occurs if one or more of the above commands have been
given out of order.

{\em refresh}
\newline
The plot refresh command will attempt to reproduce as closely as possible
the current plot, however since the refreshing is done in a strict
order (the same as for the plot all command) some differences may be
apparent (see the option "alL" for more details).

\subsection{surface-plot}
\Command{plot a surface representation of the map array}
\Parameters{catalogue-entry,bias,offset,range}

A surface representation of the current map is plotted in the graphics
overlay window or using the full plot device if graphics overlay is not
active.  Only that region of the current map specified by the SET-UV-RANGE
command is plotted.  The plot is represented as a succession of hidden-
line histograms.  Bias represents the offset (in map values) between
successive slices of the map where there is one slice for each row.
Offset is an offset in pixels to sepcify between successive rows.
Range specifies the range to plot from X_low to X_high and then
Y_low to Y_high.

The surface plot is drawn into the current scratch-plot definition and
you can control the view-port, line-style etc. for the surface plot
using the scratch-display sub-system.

\subsection{isometric-plot}
\Command{plot an isometric (joined surface mat) of the map array}
\Parameters{catalogue-entry,viewing-angle (degrees)}

An isometric plot of the specified map is made with a viewing angle
as specified -- an angle of 0 degrees views the map-data edge-on and
90.0 degrees views the map data directly from above.  This graphic
is a scratch graphic and the view-port and line colours may be controlled
with the scratch graphics subs-system.


\subsection{using_the_cursor}

\subsection{modify-lookup-table}

\subsection{cursor-position}

If you have a map drawn on the terminal (and have not used
end-plot since it was drawn), then you can find the UV, RA/DEC
and pixel value at any time by using this command. The cursor
will appear on the screen and you simply position it at the
point of interest and type any printing character (a-z, A-Z,
1-0) except 'Q' or 'q' and the information for that location
on the map will be printed on the terminal. This works very
well for devices with dual screens (Xwindows, Pericoms, Selinars,
Dacoll, or the LEXIDATA), but produces a messy result on a
BBC.





\section{The Data-Display Sub-System}

Data-display -- a sub-system for the graphical display of 1D data sets.

\subsection{Introduction
*@/mrao/anmap_v7.5/help/Intro_Data_Display.help

\subsection{line-options

\Command{setup options for each line to be displayed.
\Parameters{line option
\Options{  reset-style reset-file file x-file y-file
            x-column y-column
            x-offset y-offset x-scale y-scale x-auto-scale y-auto-scale
            x-error-bar y-error-bar x-log-scale y-log-scale
            line-style line-type error-style
            symbol-style symbol-type key-display

The options for the specified line are set.  Once a file has been defined
for a line that line will be "active" and will be displayed with a plot all
or plot line command.  The default is to use column 1 and column 2 as the
X and Y columns for the plot and to plot Y against X.  This can bechanged
with the x-column and y-column options, but you may also have the x and
y columns from different files using the x-file and y-file commands.
Error bars in X and Y may be displayed either using columns from the main
or other files (use x-error-bar y-error-bar options).  The data may be
modified "on-the-fly" before plotting using the -offset -scale -auto-scale
and -log-scale commands.   Scaling and offseting can be used to shift
and expand a line to help overlay multiple lines, the -auto-scale options
attempt to do this automatically.  The -log-scale commands take the logarithm
of each data point before display; zero or negative points are trapped and
set to zero on output.  The type of line/markers plotted is controlled
with the line-style, line-type (none, standard or histogram) and symbol-type
options, for example to plot a histogram-type line
  line-options 1 line-type 2
of to plot just symbols
  line-options 1 line-type 0
  line-options 1 symbol-type 2
to get just crosses for example.  Line-style and symbol-styles (colours
fonts size etc) are set with the line-style symbol-style commands.
Whether there is an entry in the key, and the text to go with the
entry, for a particular line is controlled with the key-display option;
the default is to include an entry in the key and use the data file name
as the text string.

\subsection{data_format
*@/mrao/anmap_v7.5/help/Spectral_data_files.help

\subsection{set-frame-style
*@/mrao/anmap_v7.5/help/Frame_Style.help

\subsection{set-text-style

\Command{set style for titles plotted in data-display graphics
\Parameters{font, size, colour, width

The standard parameters for a text style can be set with this
command -- note that the style of any frame drawn (including text)
is controlled with the set-frame-style command, but the titles are
drawn with the text-style specified with this command.

{\em attributes
*@/mrao/anmap_v7.5/help/Attributes.help

\subsection{set-title-options

\Command{set options for titles
\Parameters{title-type position outwards-displacement location justification

This command sets options for where the titles are to be displayed, to set
the text-style for titles use the set-text-style command.

  title-type     this must be one of title, x-title or y-title
  position       specifies which side of the frame to display the title
                   value   title   x-title    y-title
                     1     top     bottom     left
                     2     bottom  top        right
  outwards-displacement
                 this is the displacement of the title from the frame
                 measured in average width characters.
  location       specifies the title location relative to the frame
                 and is specified as a fractional position along the frame
                 with the left/bottom being 0.0 and the right/top 1.0 --
                 positions outside of the frame are allowed.
  justification  specifies the title justificationtion relative to the
                 location of the title and is specified as a fractional
                 position along the title with 0.0 being the left of the
                 title and 1.0 the right of the title.

{\em examples

To centre a title use location=0.5 justification=0.5
   set-title-options title 1 3 0.5 0.5

The following displays the top title aligned with the left of the frame,
the y-title is moved outward to allow space for "horizontal labels".
   set-frame-style rotate-labels
   set-title-options title 1 3 0.0 0.0
   set-title-options y-title 1 8 0.5 0.5

To put the X-title on the top of the plot and the mai title above it.
   set-title-options title 1 6 0.5 0.5
   set-title-options x-title 2 3 0.5 0.5

\subsection{set-key-options

\Command{set options for the plotting of keys
\Parameters{option
\Options{on off position frame text-style

Options for the key plotted on a data-display graph are controlled
with this command.  Whether a paticular line/set of points is included
in the key or not, and the label used in the key, is controlled with
the line-options key-display command.

The meaning of the options to the set-key-options command is:

  on              turn the display of a key on
  off             turn the display of a key off
  position        specify the position of the key relative to the
                  frame of the graph; the position is specified
                  as an x- and y-position such that 0.0 aligns the
                  left of the key with the left edge of the frame
                  1.0 aligns the left edge of the key with the right
                  edge of the frame -- for y 0.0 is the bottom and 1.0
                  is the top and the top of the frame is aligned with the
                  specified position.  For example:
                    0.0,1.0   positions the key in the top left corner
                    1.2,1.0   positions the key outside of the frame to the
                              top right.
                  It is not possible to align the key on any other edge
                  as the size of the key is not known until after it is
                  drawn as the information is contains is "dynamic".
  frame           Turn the plotting of a frame around the key on or off.
  text-style      set font, size, colour, width, drop for the text appearing
                  in the key -- it is NOT possible to control the style of
                  text for each entry in the key.  The text that is plotted
                  is controlled by the line-options key-display command.

See also: line-options

\subsection{title

\Command{set the title for the plot
\Parameters{title

Set the title for the plot to the specified text.  The title is plotted
when the frame is plotted, if the frame is already drawn you will need to
use either "initialise plot; plot all" or "plot refresh".

\subsection{x-title

\Command{set the title for the X-axis of the plot
\Parameters{title

Set the X-axis title for the plot to the specified text.  The title is plotted
when the frame is plotted, if the frame is already drawn you will need to
use either "initialise plot; plot all" or "plot refresh".

\subsection{y-title

\Command{set the title for the Y-axis of the plot
\Parameters{title

Set the Y-axis title for the plot to the specified text.  The title is plotted
when the frame is plotted, if the frame is already drawn you will need to
use either "initialise plot; plot all" or "plot refresh".

\subsection{x-range

\Command{set the x-range for the plot
\Parameters{x1 x2

Set the x-range for the plot.  If x1 and/or x2 are non-zero then the
range is set to the explicit range sepcified so that the left edge of
the plot is x1 and the right edge x2 (x2 may be < x1).  To request auto-
scaling (the default) set x1 = x2 = 0.0.

\subsection{y-range

\Command{set the y-range for the plot
\Parameters{y1 y2

Set the y-range for the plot.  If y1 and/or y2 are non-zero then the
range is set to the explicit range sepcified so that the left edge of
the plot is y1 and the right edge y2 (y2 may be < y1).  To request auto-
scaling (the default) set y1 = y2 = 0.0.

\subsection{view-port

\Command{set the (sub-)view-port for the data display, graph,  object
\Parameters{u1,u2,v1,v2

The view-port specified with this command is relative to the view-port of
the graph graphic itself (set with the graphic view-port command).  The
graph graphic is displayed with its frame aligned with this view-port,
whereas the larger view-port of the graph graphic is the region of the
display (or plot device) allocated to this graphic and if the mode is
set to opaque, the region of the screen which will be cleared.  For example
the following will ensure the entire screen is cleared when a graph graphic
is defined, but sufficient space will be allowed for the plotting of
labels and titles which are placed outside of the frame of the actual plot

   graphic select graph 1
   graphic view-port 0.0,1.0,0.0,1.0
   data-display view-port 0.15,0.85,0.15,0.85

Alternatively if the graph graphic is to appear in a small region of the
screen to the top right (perhaps allowing space for an image to be displayed)
then the following would be more appropriate

   graphic select graph 1
   graphic view-port 0.7,1.0,0.7,1.0
   data-display view-port 0.15,0.85,0.15,0.85

{\em examples

Select the whole device for display of a linear graph
   graphic
     select graph 1
     view-port 0.0,1.0,0.0,1.0
   data-display
     view-port 0.1,0.9,0.1,0.9
     linear-graph ......

Display a grey scale image then display a slice in a small box to the top right
   map-display
     set-map 1
     grey on,,,,
     plot grey
   map-analysis
     select-slice 1 -30,30,30,30
   graphic
     select graph 1
     view-port 0.7,1.0,0.7,1.0
   data-display
     view-port 0.15,0.85,0.15,0.85
     line-options 1 file anmap_results.dat
     plot all



\subsection{plot

\Command{plot a data-display graph graphic object
\Parameters{option
Options: all line frame annotations refresh

The plot command controls the plotting of information to the screen
or output device.  The plot all command can be used to plot all
the information you have requested/defined; once plotted plot all
will not re-plot the same information even if you change some
attributes such as line-colour etc. -- in this case you need to
use "plot refresh" which refreshes the plot using the attributes
for lines and text and other options in effect at the time you give
the command.  You can control in more detail how the plot is put
together using the plot line plot annotations and plot frame commands
to plot just parts of the information on the plot.  There are also
plot commands in the annotate sub-system to provide careful control
over the plotting of annotations.

The options have the following specific meanings
   all              plot everything that is outstanding (i.e. not
                    yet plotted.
   refresh          re-plot the graphic using all attributes and options
                    in effect when this command is given.
   annotations      plot annotations, same as annotate plot all or
                    plot all.
   frame            plot just the frame and key for the plot.
   line   n         plot just line n where line "n" has been defined with
                    the line-options command.

\subsection{initialise

\Command{initialise the data-display sub-system
\Parameters{option
Options: all plot options annotations title setup

This command can be used to initialise parts of the data-display
sub-system. The options have the following meaning:

   annotations     initialise all annotations and annotation options
   options         initialise line, text, key and frame style options
   title           initialise title strings
   plot            initialise for (re-)plotting, the state of the
                   sub-system is initialised so that it was as if
                   nothing had yet be drawn on the plot; plot all
                   will then plot everything.
                      initialise plot ; plot all
                   gives the same results as
                      plot all ; plot refresh
   setup           initialise the setup of line options specific to
                   the files / data sets to be plotted.
   all             equivalent to all the above


\subsection{annotate
*@/mrao/anmap_v7.5/help/Annotate.help

\subsection{get

\Command{retrieve information about the current sub-system
\Parameters{option
Options: cursor-input coordinates line frame key max-lines view-port range

Get returns information about the current scratch-display definition or
prompts the user for necessary input.  All information is returned via
command-language parameters.

   cursor-position     the user is requested to enter a position using
                       the cursor.  The routine will return
                         %X %Y       coordinate position in current graph
                                     coordinates.
                         %GR-X %GR-Y position in screen coordinates.
                         %GR-CHAR    character typed

   coordinates         return the current coordinates as an array
                         %COORDINATES
   view-port           return the current view-port as an array
                         %GRAPH-VP
   range               reruen the current data range in two arrays
                         %X-RANGE %Y-RANGE

   scratch             details of the options for the current scratch
                       definition

   max-lines           return the maximum number of lines as %MAX-LINES

   frame               return frame details in standard parameters

   key                 return key details in standard parameters

   line  n             return details for line n in standard parameters

{\em standard_parameters
*@/mrao/anmap_v7.5/help/Standard_parameters.help

\subsection{linear-graph

Command-procedure: plot a data-file as a linear-scaled graph
\Parameters{data-file

This command plots a data file as a linear-scaled graph.  This provides
a simplified interface to the facilities of the data-display system
when you have a data file in the standard spectrum format and you want
to plot the y-column (column 2) versus the x-column (column 1).  You
specify just the file name and the plot is drawn automatically.  To
alter aspects of the plot (e.g. add a title) use standard data-display
commands after the linear-graph command and the say "plot refresh".

For example to plot the anmap results file then add titles:

   linear-graph anmap_results.dat
   title Main title
   x-title X
   y-title Y
   plot refresh

\subsection{logarithmic-graph

Command-procedure: plot a data-file as a log-scaled graph
\Parameters{data-file

This command plots a data file as a log-scaled graph.  This provides
a simplified interface to the facilities of the data-display system
when you have a data file in the standard spectrum format and you want
to plot the y-column (column 2) versus the x-column (column 1).  You
specify just the file name and the plot is drawn automatically.  To
alter aspects of the plot (e.g. add a title) use standard data-display
commands after the lograithmic-graph command and the say "plot refresh".

For example to plot the anmap results file then add titles:

   logarithmic-graph anmap_results.dat
   title Main title
   x-title X
   y-title Y
   plot refresh

\subsection{examples

*


\section{The Drawing-Display Sub-System and Annotations}

Annotate -- a sub-system to annotate plots.

\subsection{Introduction

The annotate system can be used to add simple annotation-like drawing elements
to other plots or to produce more complex line-based graphics.  In addition
to a number of basic drawing commands there is also the provision for a
more complicated compound-object.

The following commands define graphical objects (the system is highly
object oriented):

  line circle ellipe arc box erase cross arrow text compound-object

while the following commands change the attributes of these objects:

  set-text-style set-line-style set-fill-style set-arrow-style
  update-object

The objects may either be plotted after a plot command (the default
action) or plotted immediately they are defined (using the set-plot-options
command).  Furthermore the coordinates for each object may be specified by
entering the coordinates in response to prompts or by using the cursor
to position objects (this is again controlled with the set-plot-options
command).  The coordinates used can either be physical device coordinates
(the default) which have the advantage of preserving the correct aspect
ratio of objects plotted among different devices, or user-defined coordinates
(see the command set-coordinates and physical-coordinates).  A few commands
exist to manipulate objects:

  plot object, delete-object, list-object, update-object

Finally the get command is provided to help in constructing command procedures
which interact with the graphics device.

The map-display, scratch-display and data-display sub-systems all have
annotate sub-systems of their own.

The coordinates used in all plotting are the corrdinates of the calling
sub-system.

\subsection{line

\Command{define a line-type graphic object
\Parameters{line-start line-end

A line object is defined using the current line style and is plotted
after a plot command or will be plotted immediately if the set-plot-options
auto-plot command has been given.  A multiple-element line can be drawn
using many line objects or using the compound-object definition.  If the
input mode is set to cursor-input simply mark each end of the line using
the cursor.

\subsection{arc

\Command{define an arc-type graphic object
\Parameters{arc-start arc-end arc-midpoint

[NOT IMPLEMENTED YET]

An arc object is defined using the current line style and is plotted
after a plot command or will be plotted immediately if the set-plot-options
auto-plot command has been given.

\subsection{ellipse

\Command{define an ellipse-type graphic object
\Parameters{centre, semi-major, semi-minor, position-angle

An ellipse object is defined using the current line style and fill mode
and is plotted after a plot command or will be plotted immediately if
the set-plot-options auto-plot command has been given.   The default
orientation for the ellipse is to have the major axis along the "x-axis",
the position angle is then the angle the major axis makes with the x-axis
measured in a positive (counter-clockwise) sense.  If the input mode is
set to cursor-input then mark the centre end of the major and end of the
minor axes.

If a filled ellipse is drawn then the colour of the filling is the same
as the outline colour.

\subsection{circle

\Command{define a circle-type graphic object
\Parameters{centre radius

A circle object is defined using the current line style and fill mode
and is plotted after a plot command or will be plotted immediately if
the set-plot-options auto-plot command has been given. If the input
mode is set to cursor input the mark the centre of the circle and
one point on the circumference.  If a filled circle is drawn then the
colour of the filling is the same as the outline colour.

\subsection{erase

\Command{define an erase-type graphic object
\Parameters{box-corner-1, box-corner-2

An erase object is defined and is "plotted" after a plot command or
will be plotted immediately if the set-plot-options auto-plot command
has been given.  You should specify (or mark in cursor-input mode) opposite
corners of the box you wish to erase -- only "box-shaped" erase regions
are currently supported.

\subsection{text

\Command{define a text-type graphic object
\Parameters{position, position-angle, 'text string'

A text object is defined using the current text style and is plotted
after a plot command or will be plotted immediately if the set-plot-options
auto-plot command has been given.  If the input mode is from the command line
then you will be asked to specify the position for the bottom left point
of the text and a position-angle for the text (measured in the positive sense,
couter-clockwise, from the x-axis); if however you are in cursor-input mode
then you mark the position of the text only -- in both input modes you will
supply the text from the command line and if the text contains spaces the
text should be quoted.

\subsection{box

\Command{define a box-type graphic object
\Parameters{box-corner-1, box-corner-2

A box object is defined using the current line style and fill mode and is
plotted after a plot command or will be plotted immediately if the
set-plot-options auto-plot command has been given. You should specify
(or mark in cursor-input mode) opposite corners of the box; the box can be
drawn in outline or filled if fill mode is set appropriately, in either case
the fill colour is the same as the outline colour.

\subsection{cross

\Command{define a cross-type graphic object
\Parameters{cross-centre arm-length

A cross object is defined using the current line style and is plotted
after a plot command or will be plotted immediately if the set-plot-options
auto-plot command has been given.   If the input mode is cursor-mode then
you mark the centre of the cross and the end of one arm.

At present crosses can only be drawn with their arms aligned with the
x and y axes.


\subsection{arrow

\Command{define an arrow-type graphic object
\Parameters{arrow-start arrow-end

An arrow object is defined using the current line and arrow styles and
is plotted after a plot command or will be plotted immediately if the
set-plot-options auto-plot command has been given.  The form of the
arrow head is controlled with the set-arrow-style command.

\subsection{set-text-style

\Command{set the text-style for subsequent objects
\Parameters{font, size, colour, width

Set the style for subsequent text -- this style can be used to update
the style of existing objects using the update-object command.

{\em attributes
*@/mrao/anmap_v7.5/help/Attributes.help

\subsection{set-line-style

\Command{set the line-style for subsequent objects
\Parameters{style, width, colour

Set the style for subsequent line drawing -- this style can be used to update
the style of existing objects using the update-object command.

{\em attributes
*@/mrao/anmap_v7.5/help/Attributes.help

\subsection{set-fill-style

\Command{set the fill-style for subsequent objects
\Parameters{fill-mode

Set the style for subsequent filled drawings -- this style can be used
to update the style of existing objects using the update-object command.
The fill style can be either:
   1 = filled
   2 = emty (outline only).

The fill colour is the same as the colour selected for the outline which
is the colour of the current line style.

\subsection{set-arrow-style

\Command{set the arrow-style for subsequent objects
\Parameters{angle-of-head, head-length, arrow-fill

Set the style for subsequent arrows -- this style can be used
to update the style of existing objects using the update-object command.
The angle of the head is the half-angle between the shaft of the arrow
and the lines forming the arrow head itself; the head length is measured
in proportion to the length of the arrow and the arrow-fill mode
(either 1 filled or 2 outline) controls how the arrow head itself is
drawn.

\subsection{coordinates

The coordinate system used in annotations is the same as the coordinate
system in use in the map-display, scratch-display or data-display
sub-system calling the annotate sub-system; this facilitates aligning the
annotations with respect to the plots.

\subsection{list-objects

\Command{list all defined graphic objects

A brief listing of all defined objects in the current annotate object is sent
to the screen.  The object status is 1 if the object is defined, but not
plotted and -1 if the object is defined and plotted.

\subsection{delete-object

\Command{delete an object from the current annotate object
\Parameters{object-number

The specified object can be deleted; note that this command does not
effect the plot until a plot refresh command is given.

\subsection{plot

\Command{plot object(s) for the current annotation
\Parameters{option
Options: all object refresh

The plot command does the actual work of plotting defined objects (but see
also the set-plot-options auto-plot command).

  plot all       plot all defined, but not plotted objects
  plot object n  plot object n only
  plot refresh   re-draw the plot, plotting only those objects already drawn,
                 this command can be used to plot the current drawing into
                 a new view-port or onto a new device.

\subsection{initialise

\Command{initialise aspects of the annotation
\Parameters{option
Options: all plot options drawing

This command can be used to initialise aspects of the annotation:

  initialise plot     initialise all objects so that they appear to be
                      "not plotted"
  initialise options  initialise options such a line, text fill and arrow
                      styles
  initialise drawing  initialise the list of objects (effectively delete all
                      objects in the annotation).
  initialise all      the combination of all of the above.

\subsection{update-object

\Command{update the style of a given object
\Parameters{object-number | all

Update the style of the specified object or "all" objects to the current
style -- this applies the current line, text, arrow, and fill styles to the
specified objects.  The effects of this command will not be appraent until
the next plot refresh command.

\subsection{set-plot-options

\Command{set options for how the plot is updated
\Parameters{option
Options: auto-plot, deferred, cursor-input, command-input

Set various options to control how the plot gets updated:

  set-plot-options auto-plot      set the plot into auto-plot state so that
                                  objects are plotted as soon as they are
                                  defined.
  set-plot-options deffered       set the plot into deffered state so that
                                  objects are not plotted until an explicit
                                  plot command.
  set-plot-options cursor-input   set the input mode to cursor input --
                                  coordinates are entered using the cursor
  set-plot-options command-input  set the input mode to command-line input
                                  -- coordinates are entered from the
                                  command line.

\subsection{get

\Command{return information for the current annotation
\Parameters{option
Options: cursor-input, coordinates, max-objects, object-type, object-status

Information for the current annotation is returned in command-language
parameters of the form "%option-name".  The options have the following
meaning:

  cursor-input       cursor input from device
  coordinates        current coordinates
  max-objects        maximum number of objects
  object-type        object type for object
  object-status      object status for object


\section{The Scratch-Display Sub-System}
\section{The graphic sub-system}
Graphic sub-system

\subsection{introduction

The graphic sub-system is used to coordinate and control all of the
graphics tools within Anmap, including the manipulation of devices and
device sizes and the allocation of regions of the device to specific
plotting tasks.  Most of the commands within the graphics sub-system
either operate on the current graphic or on the device.

All graphics-based operations and tools in Anmap use a "graphic" definition;
four basic types are currently defined:

  image      For images and contours (used by map-display)
  draw       For drawings (used by drawing-display)
  graph      For graph-like graphics (used by data-display)
  scratch    For various graphics utilities (used by various commands
                 such as plot-slice in map-analysis).

You may have more than one definition for each type which are labelled
with the graphic index number.  When Anmap starts up one graphic of each
of the above types is defined and initialised and each is given the index
number 1.  The command list-graphics can be used to list the currently
defined graphics.  Each graphic definition contains all the instructions to
produce a plot.  The commands in the graphics sub-system enable you to
manipulate graphics and their relationship to the physical device.
For all graphics some basic information is recorded:

   view-port     region on the device into which the graphic is drawn
   transparancy  the transparency of the graphic with respect to underlying
                 graphics on the device
   depth         the depth of the graphic
   line-style    a default line style
   test-style    a default text style

How the last three are used depends on the sub-system using the graphic. The
default on Anmap's startup is to set the view-port to the entire screen and
the transparency to opaque for all graphics except "draw" where the
transparency is set to transparent.

To illustrate the use of graphic definitions consider the following example
in which a map is displayed using the default graphic, it is then re-drawn
in a smaller region of the screen and another graphical object (a graph)
is drawn in another part of the screen.

! enter map display and display may number 1 as a grey scale image
  map-display
  set-map 1
  grey on,,,,
  plot grey

! now re-display this image in the upper right hand quadrant
  graphic
! first clear the device
    clear-device
! now reset the view-port for the image
    select image 1
    view-port 0.5,1.0,0.5,1.0
! return to map-display and re-plot
   map-display
     plot refresh
! enter the data-display sub-system and plot a graph having setup the view-port
   data-display
     graphic
       select graph 1
       view-port 0,.5,.5,1
     exit
     linear-graph data.dat

Note also in this example the use of "plot refresh" to update the plot.

The graphical elements can be arbitrarily complicated, all the information
is stored in the associated graphic.  Graphic definitions can also be saved
and read back into Anmap using the save and recall commands. Finally it is
possible to copy graphic definitions.

The device can be controlled with a number of commands:
  output-device     to set the output device
  open-device       open the selected device
  close-device      close the selected device
  device-size       set the physical size for the "device" (e.g.
                    size of X-window screen or plotting region on
                    paper).
  next-page         move to the next page (screen) of output.
  segment-screen    break the screen up into a number of (logical)
                    pages -- next-page will then move between them.

For example having obtained the above plot on an X-window screen the
following can be used to send the plot to a postscript file.

  graphic
    close-device
    output-device test.ps/ps
    refresh-all
    close-device
  exit

Alternatively a better approach is to use the built-in postscript- procedures:

  postscript-dump      will dump the current graphics
  postscript-view      will view the dumped file
  postscript-plot      will plot the dumped file
  postscript-save      will save the dumped file to a new file

For example the following will save the graphics as a postscript file and
then send it to the laser printer.

  postscript-dump ; postscript-print

\subsection{output-device

\Command{set the output device for subsequent graphics
\Parameters{device-name

Set the output device for subsequent graphics -- the default device is
an X-window device.  To make the device accept graphics you may need to
close any existing device; you may also open the new device using
device-open, but this will happen automatically when there is an attempt
to write to the device.

\subsection{close-device

\Command{close the current graphics device
\Parameters{NONE

Close the current graphics device -- this is necessary to flush output
to say a file containing POSTSCRIPT graphics.

\subsection{open-device

\Command{open the currently selected device
\Parameters{NONE

Open the device previously selected with the output-device command. If
a size has been specified for the device then this size will take effect.

\subsection{next-page

\Command{move the the next (logical) page
\Parameters{NONE

Move to the next logical page.  For output to a hard-copy device or a
file (e.g. a POSTSCRIPT file) a new physical page will be selected,
for output to a screen a new screen will be started; it is NOT possible
to move to the previous page.  If the device/screen has been segmented then
this commands moves to the next segment or logical page of the device,
in this case the logical pages start in the top left of the physical
device and move to the bottom right.  More control over the appearance
of the screen may be obtained by specifying the drawing area using the
view-port command to setup the drawing view-port.

\subsection{segment-screen

\Command{divide the screen up into a number of logical segments
\Parameters{number-of-horizontal-segment

The screen is divided according to the specified number of horizontal
segments, the number of segments in the vertical (y) direction is
always set to 1 or 2 and the number of horizontal segments must be less
than or equal to 3.   More control over the appearance
of the screen may be obtained by specifying the drawing area using the
view-port command to setup the drawing view-port.

\subsection{device-size

\Command{specify the physical size (in cm) for the device
\Parameters{plot-size, aspect-ratio

The device is set to the specified size in cm with the specified aspect
ratio (width/height).  A square plot is therefore obtained with aspect-ratio
set to 1.0, a "landscape" plot is obtained with aspect-ration < 1.0 and
a "portrait" plot is obtained with aspect-ratio > 1.0.  This command
can be used to match the physical size between screen and hard-copy
devices, to reduce "screen clutter" by reducing the size of the requested
graphics window, or to maximise the size of the plotted region.  Not all
devices may honour a request to specify a size to the plotting region.
To obtain the default size specify a size of 0.0 cm.

\subsection{select

\Command{select a new of existing graphic
\Parameters{graphic-name, graphic-index

A new or existing graphic definition is selected.  The graphic name must
be one of the recognized types of graphic structure:

    image      for use in map-display
    graph      for use in data display
    draw       for use in drawing-system
    scratch    for use in "scratch" plots produced by various Anmap commands.

For information on graphic structures see the introduction to the graphics
sub-system.  The graphic-index is a unique index (for each graphic definition)
which labels to definition, you are free to choose the index provided it is
an integer with a value greater than zero.  New graphics are initialised
and made current, existing graphics are simply made current -- commands in the
grahics sub-system operate on either the current device or the current
graphic.

{\em advanced_information

When a new graphic is selected then the procedure display-default is
run with parameter SUB-SYSTEM where sub-system is the appropriate name
of the sub-system to which the graphic-name applies.  For example the
command:

   select image 2

if "2" is a new image graphic will result in the following command being
executed:

   default-setup map-display

this is turn will execute a procedure to actually perform the initialisation.
Standard procedures are provided for these actions, but they can be
overiden by the user by re-defining the the display-default procedure; if
you intend to do this start with the existing definition as an example
and modify the procedure(s) as required.

\subsection{plot

\Command{plot the current graphic
\Parameters{NONE

This command is equivalent to running the plot all command in the sub-system
appropriate to the currently defined graphic.

\subsection{refresh

\Command{refresh the current graphic
\Parameters{NONE

This command is equivalent to running the plot refresh command in the sub-system
appropriate to the currently defined graphic.

\subsection{initialise

\Command{initialise the current graphic
\Parameters{NONE

This command is equivalent to running the initialise all command in the
sub-system appropriate to the currently defined graphic.

\subsection{default-text-style

\Command{setup the default text style
\Parameters{TEXT-STYLE-PARAMETERS

The text style to be used as a default by the currently selected graphic
is defined.  The way in which the default is used depends on the application.

{\em text_style
*@/mrao/anmap_v7.5/help/Text_style.help

\subsection{default-line-style

\Command{setup the default line style
\Parameters{LINE-STYLE-PARAMETERS

The line style to be used as a default by the currently selected graphic
is defined.  The way in which the default is used depends on the application.

{\em line_style
*@/mrao/anmap_v7.5/help/Line_style.help

\subsection{transparent

\Command{set the transparency of the current graphic to transparent
\Parameters{NONE

The current graphic is set to transparent -- in this mode the graphics
underlying the current graphics will not be cleared when the graphic is
drawn or refreshed -- in the standard setup this is the default for
drawings, but no other graphics.

The region of the device effected by the transparency is the view-port
of the graphic.

\subsection{opaque

\Command{set the transparency of the current graphic to opaqe
\Parameters{NONE

The current graphic is set to opaque -- in this mode the graphics
underlying the current graphics will be cleared when the graphic is
drawn or refreshed -- in the standard setup this is the default for
all graphics expect drawings.

The region of the device effected by the transparency is the view-port
of the graphic.

\subsection{depth

\Command{set the depth of the graphic
\Parameters{depth

The depth of the graphic is used to specify the order in which overlapping
graphics are re-drawn with the device-refresh procedure.

\subsection{view-port

\Command{set the view-port for the current graphic
\Parameters{u1,u2,v1,v2

The view-port for the current graphic is specified in normalized device
coordinates.  The physical region of the plot (sepcified if appropriate with
the device-size command) is mapped into the region of normalized device
coordinates:
                  0.0 <= u <= 1.0      0.0 <= v <= 1.0
This command sets the region of the plot to use for the current graphic to
a sub-region of the whole device "screen or page" by specifying the
region in terms of normalized coordinates.  All plotting will be scaled to
fit in this region and the effects of the graphic's transparency will
only be effective within the view-port of the graphic.

\subsection{list-graphics

\Command{list all the currently defined graphics
\Parameters{NONE

List all the currently defined graphics.

\subsection{save

\Command{save the current graphic to a file
\Parameters{file-name

The current graphic is saved to the specified file -- the file contains
a dump of the graphic in binary form and may not be read or edited.  The
graphic can be read back into Anmap at a later time using the recall command.

\subsection{recall

\Command{read a previously saved graphic definition
\Parameters{file-name

A graphic definition previously saved with the save command is read back into
Anmap.  The graphic type and index are the same as for the saved graphic --
if the index is the same as an existing graphic of the same type then the
existing graphic is overwritten, if not a new graphic is created.

\subsection{copy-definition

\Command{copy graphic definitions
\Parameters{from-type from-index to-type to-index

A graphic of the specified type is copied to the specified type and
index -- most usually only copies between graphics of the same type are
useful, in other cases only the basic graphic definition is usefully
copied and the new copy should be initialised for the appropriate sub-system.
The copy-definition command can be useful in conjunction with the save/recall
mechanism which does not allow the type or index of saved graphics to
be changed.

\subsection{display-graphic

Procedure: display information for graphics
Paramneters: option

Option may be one of

  current -- display information for the current graphic
  all     -- display information for all graphics

{\em advanced_information

The display command is implemented as a procedure; information is retrieved
using the get command -- users may re-define the action of the display
commmand by re-defining the procedure.

\subsection{get

\Command{get/retrieve information for graphics into parameters
\Parameters{option

The get command takes one of a number of options and information is
returned in standard parameters.

   cursor-position     return the cursor position in normalized device
                       coordinated %x, %y %gr-x %gr-y are all set by this
                       call.
   current-index       the index of the current graphic is returned in
                       %current-graphic
   current-type        the type of the current graphic is returned in
                       %current-type
   current-graphic     the sequence number of the current graphic is
                       returned in %current-graphic
   maximum-graphics    the maximum number of graphics is returned
                       in %maximum-graphics
   graphic  n          information for graphic of sequence number n
                       is returned:
                        Default line-style
                          %LINE-STATUS %LINE-STYLE
                          %LINE-COLOUR %LINE-WIDTH
                        Default text-style
                          %TEXT-STATUS %TEXT-FONT
                          %TEXT-COLOUR %TEXT-WIDTH %TEXT-SIZE
                        Graphic details for graphic n
                          %GRAPHIC-STATUS        status
                          %GRAPHIC-TYPE          type
                          %GRAPHIC-TRANSPARENT   transparency
                          %GRAPHIC-INDEX         index
                          %GRAPHIC-DEPTH         depth
                          %GRAPHIC-FILL          fill-mode
                          %GRAPHIC-VP            view-port (array)


\subsection{refresh-all

Procedure: to refresh all plotted elements on the device
\Parameters{NONE

This command can be used to refresh all plotted graphics on the
device -- in most cases it should reproduce the orginal screen
output if graphic depths have been set correctly, however a different
results will occur if you have made a lot of use of the graphic clear
commmand -- in these cases refresh each graphic element in turn.

\subsection{postscript-output

Procedures: to handle postscript output from a PGPLOT graphics device

There arefour procedures designed to simplify obtaining hard copy from
Anmap as postscript files.

  postscript-dump      will dump the current graphics
  postscript-view      will view the dumped file
  postscript-plot      will plot the dumped file
  postscript-save      will save the dumped file to a new file

For example the following will save the graphics as a postscript file and
then send it to the laser printer.

  postscript-dump ; postscript-print

With the exception of the postscript-save command none of the above commands
require any parameters; the postscript-save command can be used to save the
current dumped postscript to a permanent file.

*



\end{document}






