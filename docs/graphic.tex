\documentstyle[12pt]{article}
%
% Latex Macros for use with Anmap
%
% The Anmap Name
\newcommand{\Anmap}{{\bf Anmap}}
\newcommand{\Version}{{\bf X0.52}}
%
% Page layout suitable for an A4 printer
\topmargin 0.1in
\oddsidemargin 0.1in
\evensidemargin 0.1in
\textheight 9.0in
\textwidth 6.25in
\headheight 0in
\headsep 0in
\parindent 0in
\parskip \bigskipamount
%
%
%
% Definition of Standard LATEX environments etc.
%
% Environment definitions for numbered paragraphs:
%   npars:-      (1) ....
%   rpars:-      (i) ....
%   apars:-      (a) ....
%
\newcounter{ncount}
\newcounter{rcount}
\newcounter{acount}
\newenvironment{npars}{
   \begin{list}{(\arabic{ncount})}{\usecounter{ncount}}
                      }{\end{list}}
\newenvironment{rpars}{
   \begin{list}{(\roman{rcount})}{\usecounter{rcount}}
                      }{\end{list}}
\newenvironment{apars}{
   \begin{list}{(\alph{acount})}{\usecounter{acount}}
                      }{\end{list}}
%
%
%


\begin{document}
\begin{center}
\vspace*{4.0cm}
{\Large\bf Anmap User Guide} \\
\vspace{1.0cm}
{\large\bf to} \\
\vspace{1.0cm}
{\Large\bf Commands for Publication Graphics}
\end{center}

\newpage
\section{Introduction}

Publication quality graphics in \Anmap\ are provided for graphs
and images in a large variety of styles with control over almost
every aspect of the output.  Complicated figures can be composed
using the tools provided which include commands to layout various
aspects of the figure, annotate them and even add simple line
drawings.  The graphics facilities are {\em object oriented}, this
means that you construct plots by building up a description and
then displying it, the power of this approach comes when you want to
change an aspect of the plot, for example the line style or text size,
all you have to do is alter these aspects of the plot and then re-display,
you do not have to repeat a perhaps long and complicated series of
actions to  re-display the figure.

This Users' Guide describes the command interface to the publication
graphics facilities of \Anmap; it will therefore be of interest to
all users wanting to display such graphics and users writing scripts
which nake use of the graphics facilities.  

The graphics-related commands are organised into five sub-systems.
The {\em graphics} sub-system provides basic commands to control the
layout of graphical objects on the page, specify basic attributes for
graphical objects and control the graphic devices which \Anmap\ can send
its output to.  Each of the other sub-systems provide facilities to
construct graphic objects of one of four basic types (image, graph,
drawings and scratch graphics in the {\em map-display}, 
{\em data-display}, {\em drawing-display} and {\em scratch-display}
sub-systems respectively).  Multiple objects of each type may also be
defined all with their own (different) set of attributes.  For each
graphic object (excluding drawings) you may also specify annotations
such as text, arrows etc.

In the {\em map-display} sub-system you construct plots (graphic objects)
of two-dimensional images --- maps.  You can display the images as 
grey-scales or in false-colour, or using contours, 
vectors (for appropriate data), or symbol plots to mark specified 
data points; these various styles can be combined on a single plot.
You have full control over the style of the plot including the frame,
grid, text annotations, contour line styles, scale-bar for the grey-scale
etc.

The {em data-display} sub-system is used to create plots of 
one-dimensional data or functions where a series of ``y-values''
are specified at a series of ``x-values''.  The data may either
form a continuous graph or histogram, or they could represent points
in a scatter plot.  The data points may be annotated with error bars
in both the ``x'' and ``y'' directions, and you have full control over
the appearance of the plot deciding on whether individual points are
marked with symbols, joined by a continuous curve or a histogram-type
curve and so on.  All aspects of the frae, titles and annotations may
be specified and a key generated automatically.  On a single plot
you can specify many ``lines'' and facilities exist to provide automatic
scaling (both in y and x for comparison of say spectra) and offsetting
in y and x directions to produce a stacked-plot.  Data can be supplied
either as a file in the standard format used by \Anmap\ (it is a simple
multi-column ascii (ordinary text) file, or as a mathematical
function given in a intuative form (such as $sin(x)/x$ if you wanted to
plot a {\em sinc} function).  

The final two sub-systems, the {\em scratch-display} sub-system and 
the {\em drawing-display} sub-system provide facilities to control
the style of scratch graphics and to annotate the entire figure 
respectively.  Scratch-graphics are those graphics produced by 
various commands within \Anmap\ such as isometric or surface plots,
scatter plots or histograms of image values.  The {\em scratch-display}
sub-system provides facilities for specifying the size of such plots,
line- and text-styles.  Although each graphics sub-system provides
a set of annotation commands, there is also a {\em drawing-display}
sub-system to complement these facilities.  The annotations associated
with a particular graphic object are tied to that object (and indeed
use the same coordinate system as the object).  When you compose
a complete figure out of a number of graphic objects (for example a
number of different images), you may well wish to annotate the complete
composition perhaps with a title, or lines or arrows linking
various graphic objects; these facilities are provided for in the
{\em drawing-display} sub-system.

The remainder of this Users' Guide is organised as follows.  In the
next section commands are described which operate on the complete figure, 
or which are common to all subs-systems.  An example is then given
illustrating the basic principles of the object-oriented graphics
approach and then a more detailed description of the facilities 
available within each sub-system is given.  The aim of this guide is
to show you how to use the publication graphics, it is not an exhaustive
list of all the facilities or options available and you should consult
the on-line help documentation for a complete description of all of
the available commands.

\section{Basic Graphics Commands}

There are a number of commands which are available from all the
graphic-based sub-systems within \Anmap\, the most important of these
are the {\em plot} and {\em initialise} commands.  Since the graphic's
system is {\bf object-oriented} you construct graphics by specifying
options and when you have a complete set of options defined then you
can give a plot command.  The {\em plot} command itself requires you
to specify one of a number of options which differ between sub-systems,
but all sub-systems have at least the options {\em all} and {\em refresh}.
Similarly the {\em initialise} command requires an option and all
sub-systems support {\em all} and {\em plot}.  A simple example will
illustrate the use of these commands.

In the {\em map-display} sub-system we can can setup to display a
grey-scale representation of an image in the map-catalogue as
follows:
\begin{tabbing}
XXXXXXX \= SUB-SYSTEMS \= \kill
\> Map-Display$>$ \> {\bf set-map 3} \\
\> Map-Display$>$ \> {\bf grey on 0.0 0.15} \\
\> Map-Display$>$ \> {\bf plot grey} \\
\end{tabbing}

Nothing appears on the graphic device until the plot command is given. 
If you were to have said {\em plot all} then in addition to the grey-scale
you would have a frame, title and other annotations plotted.\footnote{The
exact output will depend on the default options specified in the system,
or your own, initialisation file}  This sequence of commands will have
drawn a scale-bar on the grey-scale with text indicating the range displayed;
to turn off the text on the scale bar you can use the following command
to change one of the many options for the scale bar:
\begin{tabbing}
XXXXXXX \= SUB-SYSTEMS \= \kill
\> Map-Display$>$ \> {\bf set-style scale-bar text-off} \\
\> Map-Display$>$ \> {\bf plot refresh} \\
\end{tabbing}
Here the {\em plot refresh} command will re-draw the grey-scale image,
but with the new option (turning off text on the scale-bar) applied. 
The command {\em plot all} would not have had any effect since it is
an instruction to plot any outshanding graphic elements not yet plotted,
whereas {\em plot refresh} is an instruction to {\bf update} the current
displayed graphic.  To illustrate these differences let us take the 
above example a step further by adding contours to the plot.  We could
try to get contours plotted using the following commands:
\begin{tabbing}
XXXXXXX \= SUB-SYSTEMS \= \kill
\> Map-Display$>$ \> {\bf linear-contours 0.02,0.02,0.14} \\
\> Map-Display$>$ \> {\bf plot refresh} \\
\end{tabbing}
This will have defined 7 contour levels, but the {plot refresh}
command will not have caused them to be plotted, instead the grey-scale
would have been re-displayed since the contours are a new element to the
plot.  To get the contours plotted we must say {\em plot contours}
({\em plot all} would also display the contours but add a frame etc, as well),
therefore the correct sequence would have been:
\begin{tabbing}
XXXXXXX \= SUB-SYSTEMS \= \kill
\> Map-Display$>$ \> {\bf linear-contours 0.02,0.02,0.14} \\
\> Map-Display$>$ \> {\bf plot contours} \\
\end{tabbing}
and subsequent {\em plot refresh} commands would plot a grey-scale
overlaid with contours.  Although this is a very simple example it
serves to illustrate the way the graphic environment within \Anmap\
is organised.

In addition to the basic commands available within all graphic-sub-systems,
their are the important commands in the {\em graphic} sub-system
itself which are used to control the overall appearance of the display
and to construct complex graphics.  It is in the {\em graphic} sub-system
that the object-oriented approach to the graphical system within
\Anmap\ becomes clear since all the available commands act on complete
graphical objects.  The commands types above to the {\em map-display}
sub-system are in fact specifying options to a graphic object of type
{\bf image} and the plot commands are performing operations to display
graphics of type {\bf image}.  Each graphical sub-system has a different
type of graphics object as listed in Table 1.
\begin{table}
\begin{tabular}{|l||l|}
\hline
& \\
{\bf Object type} & {\bf sub-system} \\
& \\ \hline
& \\
image & map-display (contours, grey-scales, vectors etc.) \\
& \\
graph & data-display (graphs, histograms, scatter-plots etc.) \\
& \\
draw & drawing-display (drawings: lines, circles, text, etc.) \\
& \\
scratch & scratch-display (various graphical output) \\ 
& \\ \hline
\end{tabular}
\caption{Graphical objects available.  Commands in the
{\em graphic} sub-system control the properties of these objects.}
\end{table}

The {\em select} command selects a particular graphical object to
which the {\em graphic} sub-system commands apply.  You may have
more than one graphical object of a given type and they are
distinguished by specifying an integer key; when \Anmap\ is
initialised one graphical object of each type is defined and is
given the numerical key of ``1'' (unity).  Therefore, to select the
{\bf image} object we have already used as an example we would type
the following:
\begin{tabbing}
XXXXXXX \= SUB-SYSTEMS \= \kill
\> Graphic$>$ \> {\bf select image 1} \\
\end{tabbing}
or to start a new {\bf image} object (retaining the original for future
use):
\begin{tabbing}
XXXXXXX \= SUB-SYSTEMS \= \kill
\> Graphic$>$ \> {\bf select image 2} \\
\end{tabbing}
where in fact the ``2'' could be any integer key chosen by us for
convenience.\footnote{Supplied procedures use graphical objects with
keys above 100, you should avoid using these or confusing effects may
result, especially for novice users.}

Having selected a particualr graphical object all subsequent {\em graphic}
commands are applied to this object.  The most important commands used
to alter the properties of a particular graphic are listed in Table 2.
\begin{table}
\begin{tabular}{|l||l|}
\hline
& \\
{\bf Command} & {\bf Action} \\
& \\ \hline
& \\
view-port & specify the view-port into which the graphic is drawn \\
& parameters: $x1$,$x2$,$y1$,$y2$ (normalised device coordinates) \\
& \\
opaque &  specify an opaque rendering for the graphic, \\
& any graphics below this object cannot be seen \\
& (default for {\bf image}, {\bf graphc} and {\bf scratch} objects \\ 
& \\
transparent & specify a transparent rendering for the graphic, \\
& graphics below this object will show through \\
& (default for {\bf draw} objects \\
& \\ \hline
\end{tabular}
\caption{Graphic sub-system commands used to alter the properties of
graphical objects.}
\end{table}

\section{A Tutorial Example}

\section{The Map-Display Sub-System}
\section{The Data-Display Sub-System}
\section{The Drawing-Display Sub-System}
\section{The Scratch-Display Sub-System}
\section{Information For Script Writers}


\end{document}





