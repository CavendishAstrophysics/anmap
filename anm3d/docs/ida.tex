\documentstyle[11pt]{article}
\input{format}
\input{pars_defn}
\begin{document}
\begin{center}
{\large \bf IDA --- A Tcl Imaging and Data Analysis extension} \\
\vspace{0.5cm}
{\large \bf Version 0.5 12 April 1995} \\
\vspace{0.5cm}
{\large Paul Alexander, Cavendish Laboratory, Cambridge, UK}\\
\end{center}

\section{Introduction}

{\bf IDA} is an image/data processing and analysis language based
on Tcl (Ousterhout 1994).  The language is implemented as a Tcl
extension and may be used with other extensions.  It provides the
tools necessary for image/data analysis, but not for display although
{\it hooks} are provided within the Tcl {\bf pgplot} extension to provide
display functions.  

This version provides support for {\em images} alone ---
in this context we consider an image to be a set of gridded data values
represented by a contnuous 4D volume.  Each element in this volume is referred
to as a pixel (for 3D data the term voxel is often used).  At each pixel
we may have a vector of data values --- therefore 5-dimensional data
may be handled.  A normal representation would be to map the 4 dimensions
of the data volume onto three spatial and one time dimension and to hold in
the vector any number of {\em physical} quantities.  These may be for example
density, pressure, temperature in a fluid-dynamical simulation, red, green
and blue for a true-colour image or components of one or more complex
quantities.  Future versions will include support for a second data format
where the data consist of a vector of values on discrete points in an
N-dimensional space and may include connectivities between the data ---
such a format is more appropriate for representing say the output of N-body
simulations or molecular data.

The internal representation is as a 5-dimension array of
4-byte floating point values; this is a necessary representation for the ---
{\bf IDA} can also be used to analyse images more usually
represented with less precision although at a cost in terms of memory usage
and for some operations processing time (but not all).
Fixing on one data type greatly simplifies
the implementation and is aimed at providing functionality rather than
the greatest efficiency.
\footnote{
It is the author's view that the power of
modern workstations is so great that many issues which were until
recently of concern --- memory usage, floating-point performance etc. --- 
are no-longer overiding considerations in the design of 
data-processing software.  Of much greater improtance is the time required
for scientists to implement relatively efficient, and accurate code.
}

The remainder of this section introduces a number of terms and concepts used
throughout the package.

\subsection{Image identifiers -- imId}

All commands acting on images take an image identifier --- in the
documentation this is abreviated to {\bf imId}.  Commands will also
return an {\bf imId} if they modify an image.
In the current implementation of the package {\bf imId}s are simply
an integer  value, in a future release this will change to be an integer
prefixed by a short string (probably {\it img}) to make the behaviour consistent
with other Tcl extensions.  You should not therefore {\it rely} on
{\bf imId}s being integers.

\subsection{Data formats}
At present three image formats are supported and
identified via a numerical {\it type}:
\begin{npars}
\item MRAO map format.
\item DDF format --- this stands for {\it Directory Data Format} and is
a very flexible format supported by the {\bf IDA} extension.  It is the
prefered format for {\bf IDA} images.
\item FITS --- Flexible image transport.
\end{npars}

Note that in the current release of the {\bf IDA} little
attention is paid to image headers beyond those items needed to read and
write data.  You should not therefore use the current implemntation to
convert between image formats and only DDF format has been tested to
any extent.  In future versions of the package there will be proper
support for image headers and for image conversion.

\subsection{Sub images}
Various commands take as some of their optional arguments a subimage.
A sub-image specifies a contiguous region of the image, for example
a range of values from a 1D image or a rectangular region in a 2D image.  
To specify the subimage you must indicate the bouding pixel in each dimension;
a standard notation, common to all imaging commands, exists to specify the
subimage by specify options in the standard Tcl manner:
\begin{verbatim}
   -x1 Ix1 -x2 Ix2 
   -y1 Iy1 -y2 Iy2 
   -z1 Iz1 -z2 Iz2 
   -t1 It1 -t2 It2 
   -v1 Iv1 -v2 Iv2
\end{verbatim}
throughout this text the term {\bf subimage} will refer to a region of the
image specified using the above notation for a sub-image.
Any optional parameters which are not supplied default to the full extent of the
image.  It is legal to specify a subimage which extends beyond the
boundary of the image --- only those pixels which are within the image
will be used in processing.

\subsection{Blanking}
All imaging commands support the concept of a {\it blank} pixel.  This is
a pixel which has been set to a particular value to indicate that it does
not contain a physically realistic number.  The value is usually chosen to
be completely out of the bounds of physically sensible values (usually
a very large negative value such as 1.0E-30).  Commands exist to blank
and un-blank regions of images.  The blanking value is usually set in the
image header or definition record for an image stored on disc.

\subsection{Normalization}
Imaging commands which sum the image make use of a normalization parameter.
The normalization parameter is usually defined in the
image header or definition record for an image stored on disc.

\section{Commands to perform basic functions on images}

\subsection{The img\_image command}
The {\bf img\_image} command is used to provide low-level access to images
and to perform image I/O.  The synopsis for the command is:
\begin{verbatim}
   img_image option ?parameters?
\end{verbatim}
Table 1 provides a summary of the available options.
\begin{table}
\begin{tabular}{|l||l|}
\hline
& \\
{\bf Options} & {\bf Notes} \\
& \\ \hline
read ?-file {\bf file}? ?-type {\bf type}? &
\begin{minipage}[t]{11.0cm}
Read a data file from {\bf file}.  The format of the data is specified
by the parameter {\bf type} and has one of the values listed in Section 1.2.
\end{minipage} \\
& \\
\begin{minipage}[t]{4.0cm}
write ?-imid {\bf I}? ?-file {\bf file}? ?-type {\bf type}?
\end{minipage} &
\begin{minipage}[t]{11.0cm}
Read a data file from {\bf file}.  The format of the data is specified
by the parameter {\bf type} and has one of the values listed in Section 1.2.
\end{minipage} \\
& \\
destroy {\bf imId} & destroy the data structure associated wtih {\bf imId}\\ 
& \\ 
create {\bf defn} &
\begin{minipage}[t]{11.0cm}
create a new image with dimensions given by {\bf defn} which is a
normal Tcl array which should have all the elements required for a
definition record.  One method of ensuring this would be to read the
definition record from another {\bf imId} and modify it.
\end{minipage} \\ 
& \\ 
collapse {\bf imId} & 
\begin{minipage}[t]{11.0cm}
reduce dimansionality of {\bf imId} by collapsing any dimension of unity.  
This option sets range on dimensions of unity to 1.
\end{minipage} \\ 
& \\ 
next & return next free image ID {\bf imId}\\ 
& \\ 
exists {\bf imId} &
\begin{minipage}[t]{11.0cm}
check whether {\bf imId} is valid and exists.  The command returns a
boolean indicating true (1) if {\bf imId} exists.
\end{minipage} \\
& \\ 
max & return maximum number of images that may be allocated.\\ 
& \\ 
defget {\bf imId} {\bf defn} &
\begin{minipage}[t]{11.0cm}
get the definition record, returned in array {\bf defn}
\end{minipage} \\ 
& \\ 
defset {\bf imId} {\bf defn} &
\begin{minipage}[t]{11.0cm}
set the definition record for {\bf imId} from the array {\bf defn}.
\end{minipage} \\ 
& \\ 
defread {\bf file} {\bf defn} & 
\begin{minipage}[t]{11.0cm}
read the definition record for file {\bf file};
the definition record is returned in array {\bf defn}.
\end{minipage} \\ 
& \\ 
defputs {\bf imId} & display the definition record for {\bf imId} on stdout\\ 
& \\ 
& \\ \hline
\end{tabular}
\caption{Options and return values for the {\bf img\_image} command.}
\end{table}
The most important use for the {\bf img\_image} command is to perform
I/O for images. 
An example illustrates the use of the {\bf img\_image} command to
read in data and display it using the {\bf pgplot} extension which
defines a new command {\bf pg}.
\begin{verbatim}
   set n [img_image read -file ex1.dat -type 1]
   pg stream open 1 /xwindow
   set st [img_anal $n statistics]
   pg image $n [lindex $st 0] [lindex $st 1]
\end{verbatim}
In this example the {\bf img\_anal} command (Section 2.2) is used to annalyse 
the image and obtain statistics which are needed for {\bf pg image} --- 
this command takes arguments which are the 
minimum and maximum of the range to display.

\begin{table}
\begin{tabular}{|l||l|}
\hline
& \\
{\bf Options} & {\bf Notes} \\
& \\ \hline
\begin{minipage}[t]{4.0cm}
\flushleft 
statistics 
?-gate $g$? ?-agate $ag$?
?-min $m1$? ?-max $m2$?
\end{minipage} &
\begin{minipage}[t]{11.0cm}
obtain statistics for an image. Only pixels satisfying:
\[
val > g  \; \&\&  \; abs( val ) > ag \; \&\&  \; m1 \le val \le m2
\]
are analysed in the specified subimage.  
A list of \{max min mean sd min-pixel max-pixel\}
is returned.  The defaults are such as to analyse all pixels in the
image.
\end{minipage} \\
& \\ 
\begin{minipage}[t]{4.0cm}
\flushleft 
maxmin 
?-gate $g$? ?-agate $ag$?
?-min $m1$? ?-max $m2$?
\end{minipage} &
\begin{minipage}[t]{11.0cm}
obtain the maximum and minimum for an image. Only pixels satisfying:
\[
val > g  \; \&\&  \; abs( val ) > ag \; \&\&  \; m1 \le val \le m2
\]
are analysed in the specified subimage.  
A list of \{max min min-pixel max-pixel\}
is returned.  The defaults are such as to analyse all pixels in the
image.
\end{minipage} \\
& \\ 
\begin{minipage}[t]{4.0cm}
\flushleft 
sum 
?-gate $g$? ?-agate $ag$?
?-min $m1$? ?-max $m2$?
\end{minipage} &
\begin{minipage}[t]{11.0cm}
obtain the sum of pixel intensities for an image. Only pixels satisfying:
\[
val > g  \; \&\&  \; abs( val ) > ag \; \&\&  \; m1 \le val \le m2
\]
are summed.  The sum of pixel intensities satisfying the above
constraints and within the specified subimage is returned scaled by the
normalization parameter.  
The defaults are such as to analyse all pixels in the image.
\end{minipage} \\
& \\ 
\begin{minipage}[t]{4.0cm}
\flushleft 
histogram ?-ndata {\bf ndata}? 
?-low {\bf low}? ?-high {\bf high}?
\end{minipage} &
\begin{minipage}[t]{11.0cm}
Extract a 1D image which is a histogram of image data in the specified 
sub-image of the image. The command returns the imId of the new image 
which has xdim = {\bf ndata} and vdim=2; the first vector argument is 
the value for the bin,
the second is the number of pixels within the bin.  This format is appropriate
for display using the {\bf pg} extension. The default for {\bf ndata} 
is 100 and {\bf low/high} default to the min/max on the image.
\end{minipage} \\
& \\ 
& \\ \hline
\end{tabular}
\caption{The {\bf img\_anal} command for basic image analysis.  
The first argument to
{\bf img\_anal} must be a valid {\bf imId} and the second one of the above 
options.  All the options take as an additional argument a {\bf subimage}
specification to limit the analysis to a specified sub-region of the
image.  The {\bf subimage} specification should occur after all
compulsary arrguments.}
\end{table}

\subsection{Image analysis with img\_anal}

Analysis of images is performed using the {\bf img\_anal} command.
The synopsis for the command is:
\begin{verbatim}
   img_anal imId option ?parameters? ?subimage?
\end{verbatim}
Options for the command are listed in Table 2.  For all forms of the command
you may specify a {\bf subimage} (see section 1.2) which should appear
after all compulsary arguments; only pixels within the specified subimage
are analysed.
The following example illustrates how the output of the {\bf statistics}
option can be used to implement an algorithm which attempts an
analysis of noise in an image by assuming highly discrepant points are in
fact true signal.
\begin{verbatim}
   proc img_noise { imId nSD} {
    # determine mean and SD of image data
      set st [img_anal $imId statistics]
    # discard points more than nSD standard deviations from the mean 
    # and recalculate the statistcis for the image.
    # Iterate three times
      for {set n 0} {$n < 3} {incr n} {
         set min [expr [lindex $st 2] - $nSD * [lindex $st 3]]
         set max [expr [lindex $st 2] + $nSD * [lindex $st 3]]
         set st [img_anal $imId statistics -min $min -max $max]
      }
      return [lindex $st 3]
   }
\end{verbatim}

\subsection{Modifying images with img\_apply}

The command {\bf img\_apply} takes an image and modifies it according to
a series of options which are applied sequentially to the image.  The
first agument to the command must be an {\bf imId} and subsequent arguments
are modifier options which themselves may take additional arguments as
either parameter values of other {\bf imId}s of other images.  The range
of options available with the {\bf img\_apply} command is given in 
Tables 3 and 4.
\begin{verbatim}
   img_apply imId ?subimage? ?option ?parameter?? ?option ?parameter?? ...
\end{verbatim}
A {\bf subimage} may be specified which limits the command to operate on the
specified subimage --- the subimage may be specified anywhere on the
command line provided it does not break up an option which has to be followed
by a specified set of parameters; for clarity it is recommended that the
sub-image specification follows directly after {\bf imId}.
\begin{table}
\begin{tabular}{|l||l|}
\hline
& \\
{\bf Options} & {\bf Notes} \\
& \\ \hline
= {\bf imId1} & 
\begin{minipage}[t]{11.0cm}
set {\bf imId} equal to {\bf imId1}; the operation is applied
only to the {\bf subimage} common to both images.
\end{minipage} \\
& \\
== {\bf imId1} & 
\begin{minipage}[t]{11.0cm}
set {\bf imId} equal to {\bf imId1}; the operation is applied
to the {\bf subimage} of {\bf imId}.  Values are taken sequentially
from {\bf imId1} such that the first pixel in
the subimage of {\bf imId} is set equal to the first value of {\bf imId1}.
\end{minipage} \\
& \\
+=, -=, *=, /=, {\bf imId1} &
\begin{minipage}[t]{11.0cm}
As for ``='' except:
\newline
{\bf imId} = {\bf imId} {\bf op} {\bf imId1}, where {\bf op} is +, -, *, / for
addition subtraction multiplication and division respectivley.
\end{minipage} \\
& \\
+==, -==, *==, /==, imId1 &
\begin{minipage}[t]{11.0cm}
As for ``=='' except:
\newline
{\bf imId} == {\bf imId} {\bf op} {\bf imId1}, where {\bf op} is +, -, *, / for
addition subtraction multiplication and division respectivley.
\end{minipage} \\
& \\
atan2 {\bf imId1} {\bf imId2} & set {\bf imId} = atan2( {\bf imId1}, {\bf imId2} ) \\
& \\
& \\ \hline
\end{tabular}
\caption{Options for the img\_apply command which combine images.  
The first argument to img\_apply must be a valid {\bf imId} and subsequent 
arguments are options as specified in this and the next table applied 
sequentially.}
\end{table}

\begin{table}
\begin{tabular}{|l||l|}
\hline
& \\
{\bf Options} & {\bf Notes} \\
& \\ \hline
\begin{minipage}[t]{4.0cm}
log, ln, log10, exp, sqrt
sin, cos, tan, asin, acos, atan 
\end{minipage} &
\begin{minipage}[t]{11.0cm}
Apply specified function to imId:
\newline
set {\bf imId} = func( {\bf imId} )
\end{minipage} \\
& \\
10 & set {\bf imId} = 10**{\bf imId} \\
& \\
raise {\bf x} & raise {\bf x} to power {\bf imId}; 
set {\bf imId} = {\bf x}**{\bf imId} \\
& \\
pow {\bf x} & raise {\bf imId} to power {\bf x}; set {\bf imId} = {\bf imId}**{\bf x}\\
& \\
scale {\bf x} & scale {\bf imId} by {\bf x}; set {\bf imId} = {\bf x} * {\bf imId}\\
& \\
offset {\bf x} & offset {\bf imId} by {\bf x}; set {\bf imId} = {\bf imId} + {\bf x}\\
& \\
binary {\bf x} & force {\bf imId} to be a binary image:\\
& if {\bf imId} $<$ {\bf x} then set {\bf imId} = 0.0, else set {\bf imId} = 1.0 \\
& \\
set {\bf x} & set {\bf imId} to the value {\bf x} \\
& \\
pass {\bf x1} {\bf x2} {\bf xI} & pass only values in interval {\bf x1} to {\bf x2}:\\
& if  {\bf imId} $<$ {\bf x1} or {\bf imId} $>$ {\bf x2} then set {\bf imId} = {\bf xI} \\
& \\
range {\bf x1} {\bf x2} {\bf op} {\bf xI} &
\begin{minipage}[t]{11.0cm}
For values in the interval {\bf x1} to {\bf x2} apply the specified operation.
{\bf op} should be one of =, +, $-$, * or /. Then:
\newline
if {\bf x1} $<$ {\bf imId} $<$ {\bf x2}  then set {\bf imId} = {\bf imId} {\bf op } {\bf xI} 
\end{minipage} \\
& \\
quantize {\bf x1} {\bf x2} {\bf xI} & quantize image to values are multiples of\\
& {\bf xI} in the range {\bf x1} to {\bf x2}\\
& \\
blank & set image to blank values\\
& \\
unblank {\bf x} & set all blank pixels to value {\bf x}\\
& \\
inv & invert the image; {\bf imId} = 1.0/{\bf imId}\\
& \\
& \\ \hline
\end{tabular}
\caption{The img\_apply command for modifying
images.  The first argument to
img\_apply must be a valid {\bf imId} and subsequent arguments are options
as specified in this table applied sequentially.}
\end{table}

For example the following command averages four images:
\begin{verbatim}
   img_apply 0 += 1 += 2 += 3 scale 0.25
\end{verbatim}

In the following example we quantizes an image into a set of integer levels
and then displays each level in turn.  To do this a temporary copy of the
image is taken so that this can be used as work space:
\begin{verbatim}
   # quantize the image --- note the variable img1 just contains the the imId 0
   set img1 [img_apply 0 quantize 0.0 100.0 1.0]

   # make a copy for use as workspace
   set img2 [img_geom $img1 subimage]

   # loop and display all level using the img_display procedure
   # note how img2 is constantly reset to img1
   for {set n 0} {$n < 100} {incr n} {
      set m [expr $n + 1]
      img_apply $img2 = $img1 pass ${n}.0 ${m}.5
      img_display $img2
   }
   img_image destroy $img2
\end{verbatim}
This loop in this example could have been written very compactly as:
\begin{verbatim}
   for {set n 0; set m 1} {$n < 100} {incr n; incr m} {
      img_display [img_apply $img2 = $img1 pass ${n}.0 ${m}.5]
   }
\end{verbatim}
Note that at the end of the operation we destroy the temporary image
{\bf img2} --- in a long program without carefully destroying intermediate
images the memory usage will rise quickly and you will eventually use up all
the available image descriptors.

\subsection{Geometrical transformations with img\_geom}

The {\bf img\_geom} command is used to perform geometrical transformations
on images.  Some of the operations result in new images others are
performed ``in place'' --- if the operation is stricly reversible then the
latter method is usually followed.

Table 5 lists the options to the {\bf img\_geom} command --- the first
argument to {\bf img\_geom} is the {\bf imId} of the image to transform
the second is the tranformation option as listed in Table 5.

\begin{table}
\begin{tabular}{|l||l|}
\hline
& \\
{\bf Options} & {\bf Notes} \\
& \\ \hline
subimage {\bf subimage} &
\begin{minipage}[t]{11.0cm}
Extract a subimage for {\bf imId}.  The format of a subimage is described in
section 1.3.  A new image is created by this command and the {\bf imId} of the
new image is returned.
\end{minipage} \\
& \\
project axis &
\begin{minipage}[t]{11.0cm}
Project the image along the specified axis.  The {\bf axis} argument
should be one of {\bf x, y, z, t} or {\bf v}.  This reduces the
dimensionality of the image by one.  The projection is accomplished
by summing all pixels to be projected onto a given direction.  If
any data selection is required (gating of the data for example) this
should have already been performed.
\end{minipage} \\
& \\
\begin{minipage}[t]{4.0cm}
\flushleft
bshift 
?-dx {\bf x}?
?-dy {\bf y}?
?-dz {\bf z}?
?-dt {\bf t}?
\end{minipage} &
\begin{minipage}[t]{11.0cm}
Perform a barrel shift on the image with pixel offsets given
x, y, z, t.  The defaults for the pixel offsets are 0 in all directions.
A barrel shift shifts the image by an integer number of pixels in all
dimensions and is performed using ``periodic boundary conditions'' so that
pixels projected outside the image will be wrapped back onto the opposite
side of the image.
\end{minipage} \\
& \\
flip axis &
\begin{minipage}[t]{11.0cm}
Flip the image in the direction specified by axis.
The {\bf axis} argument should be one of {\bf x, y, z, t}.
\end{minipage} \\
& \\
transpose axis-list &
\begin{minipage}[t]{11.0cm}
Transpose the axes of the image.  The new order of the axes
should be listed after the transpose option.  For example the
following will transpose the x and y axes for an X-dimensional image:
\begin{verbatim}
   img_geom 0 transpose y x
\end{verbatim}
\end{minipage} \\
& \\
\begin{minipage}[t]{4.0cm}
\flushleft
slice {\bf subimage}
\newline
?-ndata {\bf ndata}?
\end{minipage} &
\begin{minipage}[t]{11.0cm}
Extract a slice from the image.  The subimage format is used to
specify the extent of the slice which is taken from (x1, y1, z1, t1, v1)
to (x2, y2, z2, t2, v2).  The data are sampled along this slice to
give {\bf ndata} points.  The slice is returned as a 1D image with
vdim set to 2; the first vector argument is the length along the slice,
the second is the interpolated image value.  This format is appropriate
for display using the {\bf pg} extension.  {\bf ndata} defaults to 100.
\end{minipage} \\
& \\
& \\ \hline
\end{tabular}
\caption{The img\_geom command for performing geometrical
transformations to images.  The first argument to the command is
the {\bf imId} of the image to transform, the second is the option
as listed in this table.}
\end{table}

\begin{table}
\begin{tabular}{|l||l|}
\hline
& \\
{\bf Options} & {\bf Notes} \\
& \\ \hline
\begin{minipage}[t]{4.0cm}
\flushleft
transform 
\newline
?-matrix \{{\bf matrix-elements}\}?
\newline
?-vector \{{\bf vector-elements}\}?
\newline
?-cvector \{{\bf cvector-elements}\}?
\newline
?-m11 {\bf m11}? ... ?-m44 {\bf m44}?
\end{minipage} &
\begin{minipage}[t]{11.0cm}
Apply a generalised linear transform to the image.  The transform
is specified in terms of a matrix and vector whose elements may be
real, floating-point, values.  The transformation is applied to determine
the mapping of the pixels of the new output image (the {\bf imId} of which is
returned by this command) to those of the original image.  The number
of elements supplied for the matrix must correspond to a square matrix
and should form a propoer Tcl list.  The corresponding matrix is taken
as the upper left sub-matrix of the full 4D tranformation matrix.
If the matrix is $M$, the vector $V$ and the c-vector $C$, then
pixels in the output image $po$ to the input image $pi$ are related by:
\[
po = V \; + M.(pi \; - \; C) \; + C
\]
therefore $V$ provides a linear translation and $M$ a general tranformation
about a point $C$.
\end{minipage} \\
& \\
& \\ \hline
\end{tabular}
\caption{The img\_geom command for performing generalised
transformations to images.  The first argument to the command is
the {\bf imId} of the image to transform, the second is the option
as listed in this table.}
\end{table}

\end{document}
